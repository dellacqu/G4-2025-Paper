%%%%%%%%%%%%%%%%%%%%%%%%%%%%%%%%%%%%%%%%%%%%%%%%%%%
% biasreverese.tex
%%%%%%%%%%%%%%%%%%%%%%%%%%%%%%%%%%%%%%%%%%%%%%%%%%%
\label{sec:biasreverse}

\subsection{\textbf{Biasing and reverse Monte Carlo}}\label{sec:biasrev}
There is a class of applications for which rare events are of interest, and for
% \footnote{``Event'' has to be understood in a generic sense here: 
% this is not a \gclass{G4Event}, but can be any situation of interest.}
which standard, or analog, simulation is inefficient, or even impractical.
% Note that in this context ``event'' should be understood as any occurrence of
% interest and not a \gclass{G4Event}.
Estimating shielding efficiency in a radioprotection problem is one example: an
analog simulation would spend by far most of its time transporting numerous 
particles and interaction products inside the shield, with only a tiny fraction
of particles leaving the shield.  In this case, these latter particles are the 
``rare events'' (note that in this context ``event'' should not be understood 
as a \gclass{G4Event}).  At the opposite extreme would be the simulation of very
thin detectors, which is also difficult as only a small fraction of particles 
may interact inside.  Another example is the simulation of single event upsets 
in an electronic chip within a satellite.  Here the volume of interest is very 
small compared to the overall structure in which the simulation is to be 
performed.  The events of interest (track steps in the small volume) which 
deposit energy in the chip are rare given that they occur in a fraction of time
which is of the order of the ratio of the electronic chip volume to the overall 
satellite volume.

Event biasing techniques attempt to address these problems by magnifying the 
occurrence of the rare events.  This change in the simulation is reflected in
the statistical weight associated with each particle and in the ratio of 
probabilities in the biased and analog schemes to observe the particle in the
current state.  This approach allows the simulation efficiency to be boosted 
with respect to the rare events without sacrificing the physics quality 
provided by the full, detailed Monte Carlo description.

There is a large variety of biasing techniques, but they rely mostly on two main
approaches: splitting and killing (SK) \cite{bib:SK}\cite{bib:biasGeneral} and 
importance sampling (IS) \cite{bib:ImpS}\cite{bib:biasGeneral}.

In the first approach, tracks are split (killed) as long as they approach
(recede from) the phase space region of interest.  The shielding problem can be 
addressed as follows: to compensate for the natural absorption in the shield 
material and allow a sizeable number of particles to exit the shield, the flux
is artificially regenerated by splitting particles into $n$ identical copies at
regular intervals for particles moving forward, that is, on their way toward 
exiting the shield.  At each splitting, the particle weight is divided by $n$. 
If the particles move backward, they are randomly killed with probability $1/n$,
and their weights are multiplied by $n$ if they survive.  This maintains a 
high unweighted flux, leading to workable statistics, while the weighted flux
is low, as physically expected.  In this SK approach, the physics processes
are kept unchanged.

In importance sampling, the underlying physical laws are substituted with biased
laws that favor the occurrence of rare events.  The case of a thin detector 
volume can be treated as follows: the natural exponential law is replaced by a
truncated one, limited to the track path inside the volume in order to guarantee
that the collision will occur there.  The weight acquired by the particle, or 
its interaction products, when such a forced collision occurs is the ratio of 
the two interaction probability law values at the interaction point.  Thus the 
interaction law of a physics process is modified.  The final state production 
law of the process may also be biased.

Biasing can be applied at various levels in a simulation application.  Particle
generator biasing is a simple case: the analog generator is biased to favor 
phase space regions or channels of interest.  Such generator-level biasing does 
not necessarily require the weight to be propagated through the simulation stage.
The production of rare particles in high energy physics or the generation of 
specific decay chains are examples of this case, with re-weighting of the rare 
event being applied at the analysis stage.  Such generator-level biasing is of
the IS type.  When biasing is applied after the generation stage, at tracking 
time, weight computation must be handled by the simulation engine, as it will 
evolve during the particle transport.

The biasing functionalities of \Gfour{} are presented here.  At the time of the 
previous \Gfour{} general paper \cite{bib:generalpaper2} there were three main 
biasing techniques available:
\begin{itemize}
\item the Generalized Particle Source (GPS) \cite{bib:AppDevGuide}, a versatile
      set of user commands to generate primary projectiles with various weights,
      spectra and particle types; 
\item geometry splitting and weight window \cite{bib:AppDevGuide}, a method in 
      which the user specifies weight bounds for each space-energy cell of the 
      geometry, with tracks split or killed according to these bounds;
      with an extension of the scheme \cite{detmodeling:parGeom} to handle
      cells in parallel geometries, each of these being assignable to a given
      particle type;
\item hadronic cross section biasing, which allows an overall scale factor to be
      applied to cross sections, with corresponding corrections to secondary 
      particle weights, for a few relatively rare processes. 
\end{itemize}

Since then the reverse Monte Carlo technique was introduced and work began to
extend biasing functionalities using a generic approach.  Both of these efforts
are discussed below.

\subsubsection{Reverse Monte Carlo}
Reverse Monte Carlo is a biasing technique in which the primary particles 
produced by the event generator are tracked backwards in time, acquiring 
energy along the way, to determine their statistical weight.  This technique is
especially useful in the simulation of the dose received by a small electronic
chip placed inside the large, complex structure of a satellite.  The actual 
particle source consists of highly energetic electrons, and/or protons in the 
space environment, generally modeled in the simulation as a sphere that envelops
the satellite.  In this situation the fraction of simulation steps which 
contribute to the dose in the chip for particles starting from this source is of
the order of the ratio of the chip volume to the satellite volume, and hence is
very low.

In the reverse Monte Carlo approach particle tracking begins in the vicinity of
the volume of interest, using an arbitrary distribution.  In the first stage, 
primary particles are tracked backward in time, with time-reversed physics 
processes.  Such processes and particles are referred to as ``adjoint''.  At 
each interaction, an adjoint particle acquires energy and may also change its 
type: an adjoint $e^-$ may, for example, become an adjoint $\gamma$ if an 
adjoint photo-electric process takes place.  Adjoint particles are tracked back
to their extended source.  At the end of this tracking the weight of the adjoint
particle represents the statistical weight that the full reverse track (from the
adjoint source to the external source) would have in a forward simulation.  In 
other words it represents the probability belonging to this specific track from 
the external source to the adjoint source.

The arbitrary spectrum chosen in the vicinity of the volume of interest leads 
to a biased spectrum at the source.  The statistical weight of a primary 
particle is then given by the ratio of the actual to the biased distribution 
values at the source.  Note that this weight may be zero, leading to the 
particle being ignored, if the adjoint particle has an energy which is outside 
the energy bounds of the source.  Once these weights are determined, the second 
stage begins in which particles are tracked with the usual, time-forward physics
processes, beginning from the same space-time point where the adjoint transport 
started.  The weights obtained during the reverse tracking stage allow a proper
accounting of their contribution to the dose.

\paragraph{Design}
The implementation of reverse Monte Carlo in \Gfour{} \cite{bib:revMC} was 
designed to reduce as much as possible the modifications of user code required
in order to use it.  The 
\gclass{G4Adjoint\allowbreak{}Simulation\allowbreak{}Manager} class should be 
instantiated in \texttt{main()} and the physics list should be adapted in order
to create adjoint particles (electrons, protons and gammas), and to declare 
their corresponding adjoint electromagnetic processes.  An example of such a 
physics list is given in \verb"examples/extended/biasing/ReverseMC01".

During an event in reverse simulation, pairs of adjoint and forward equivalent 
particles (same type and energy but opposite direction) are generated randomly
at the same position on a surface set by the user and containing the small 
sensitive volume considered in the simulation.  This is called the adjoint 
source.  The reverse tracking and computation of the weight of the adjoint 
particle is controlled by the reverse machinery and no user code is required
for the control of the tracking phase.  During the forward tracking phase of 
the equivalent normal particle the usual \Gfour{} actions coded by the user are 
applied in order to compute the various tallies of the simulation. 

The analysis part of the user code should be modified to normalize the signals
computed during the forward tracking phase to the weight of the equivalent
adjoint particle that reaches the external surface.  This weight represents the
statistical weight that the full reverse track (from the adjoint source to the 
external source) would have in a forward simulation.  If a forward-computed 
signal is multiplied by the weight of the reverse track and is registered as a
function of energy and/or direction it will give the response function of the 
signal.  To normalize it to a given spectrum it has to be further multiplied by
a directional differential flux corresponding to this spectrum.  The weight, 
direction, position, kinetic energy, and type of the last adjoint particle that
reaches the external source, and that would represent the primary of a forward
simulation, can be obtained by public methods of the 
\gclass{G4Adjoint\allowbreak{}SimManager}
class.  More details on how to adapt user code to use the reverse Monte Carlo 
scheme are provided in the Application Developer Guide \cite{bib:AppDevGuide}.

\paragraph{Performance}
The performance of the reverse Monte Carlo can be evaulated using the execution
time and relative precision of the results.  A useful figure of merit ($FOM$) is 
\begin{equation}
FOM = \frac{1}{R^2 T} , 
\end{equation}
where $R$ is the relative precision reached for a given execution time $T$, in
minutes.  For a typical application \cite{bib:revMC}, FOM is 4.9 for a forward
method, compared to 7600 for a reverse method.  This corresponds to a time 
speed-up factor of 1250.  Such results will of course vary depending on set-up
and physics, but it is clear that with this method, large speed-up factors can 
be achieved without sacrificing precision.    

\subsubsection{Generic Biasing}
In an attempt to unify the various forward-tracking biasing techniques, a new
scheme was introduced in release 10.0.  It aims to provide the user, through a
restricted set of base classes, flexible means to handle virtually any type
of biasing.

\paragraph{Design}
The design relies on two main abstract classes.  
\gclass{G4\allowbreak{}V\allowbreak{}Biasing\allowbreak{}Operation}
represents any type of biasing operation: splitting/killing or physics process 
modification (of the interaction law, or of the final state generation).  The 
second class, \gclass{G4VBiasingOperator}, is the decision-making entity, and 
selects the biasing operations to be applied during the current step.

A third, concrete class is \gclass{G4Biasing\allowbreak{}Process\allowbreak{}Interface}
which derives from \gclass{G4VProcess} and provides the interface between the
tracking and the biasing.  At tracking time,
\gclass{G4Biasing\allowbreak{}Process\allowbreak{}Interface} checks for a possible 
\gclass{G4VBiasing\allowbreak{}Operator} associated with the current volume.  If
such an operator exists, \gclass{G4Biasing\allowbreak{}Process\allowbreak{}Interface} 
requests the operator for biasing operations to be applied.  If no operation is 
returned, \gclass{G4Biasing\allowbreak{}Process\allowbreak{}Interface} continues
with standard tracking.

A \gclass{G4Biasing\allowbreak{}Process\allowbreak{}Interface} object can wrap a
physics process so that it controls it, and takes over the standard behavior in 
volumes where biasing is applied.  At the beginning of the step the current 
\gclass{G4VBiasing\allowbreak{}Operator} may request the 
\gclass{G4Biasing\allowbreak{}Process\allowbreak{}Interface} to apply an 
occurrence biasing operation to the physics process.  The operator may also
request \gclass{G4Biasing\allowbreak{}Process\allowbreak{}Interface} to apply a
final state biasing operation when \gclass{Post\allowbreak{}Step} process 
actions are invoked.  If a \gclass{G4Biasing\allowbreak{}Process\allowbreak{}Interface}
object does not wrap a physics process, it is meant for applying splitting/killing
biasing operations.

For the case of occurrence biasing, the problem is specified by providing
a biased interaction law, represented by the abstract class 
\gclass{G4VBiasing\allowbreak{}Interaction\allowbreak{}Law}. This is discussed
in more detail below. 

\paragraph{Occurrence biasing case}
\Gfour{} transports particles by allowing discrete interaction physics processes
to compete in producing the shortest distance to the next interaction.  Each 
distance is sampled according to an exponential law driven by the process 
``interaction length'' (inverse of cross section);  the competition makes the 
sampling equivalent to one that would be made using the total cross section.
Occurrence biasing consists in substituting some other interaction law for the
analog exponential one.  The exponential transform \cite{bib:exptran} and 
forced collision are of this type of biasing.

The weight computation relies on the following formalism.  For $N(\ell)$ 
particles present at distance $\ell$, and in the asymptotic limit of 
$N(\ell)\to \infty$, the positive number $-{\rm d}N$ of these interacting
in the next segment ${d}\ell$ is related to the cross section 
$\sigma(\ell) \ge 0$ at this position by
\begin{equation}
\label{eqn:bias_rel_loss}
\frac{-{\rm d}N}{N(\ell)}=\sigma(\ell) {{\rm d}\ell}.
\end{equation}
The quantity $\sigma(\ell) {{\rm d}\ell}$ is the one-particle probability to 
undergo an interaction in the segment $[\ell, \ell+{\rm d}\ell]$.  Note that
$\sigma(\ell)$ is not assumed to result from physical cross sections per nucleus
but rather that it is a simple function of $\ell$, referred to here as an
``effective cross section''.  Making $-{\rm d}N$ proportional to $N(\ell)$
in~(\ref{eqn:bias_rel_loss}) assumes that tracks are independent of each other.
Also, only a dependence on $\ell$, and on no previous coordinates, is considered
in~(\ref{eqn:bias_rel_loss}) in order for $N$ and $\sigma$ to describe the 
interaction probability in the next segment ${d}\ell$; this assumes the 
transport is of Markov nature.  Our biasing scheme is based on these two 
fundamental assumptions.

%For a neutral particle in a uniform medium, the physical cross section is 
%independent of $\ell$, and $\sigma(\ell) \equiv \sigma^{\rm neutral}_\phi$, 
%where $\sigma^{\rm neutral}_\phi$ is the macroscopic cross section in this medium.
%Its value changes only after an elastic interaction.  In contrast, as will be shown
%below, a biased scheme generally leads to an effective cross section which
%depends on $\ell$ and changes its value even in the absence of an interaction.

%The case of a charged particle in a uniform medium is more delicate as the 
%genuine dependence of the cross section is on the particle energy.  The particle
%energy evolves during the transport between collisions because of energy loss,
%which is the cumulative effect of a long series of small elastic (in the center
%of mass) interactions, or because of acceleration by an electric field, for
%example.  Thus $\sigma(\ell) \equiv \sigma^{\rm charged}_\phi(E(\ell))$.  One 
%complication is that $E(\ell)$ is not a unique function, valid for all particles
%initially in the same state at $\ell = 0$, but has fluctuations due to each 
%particle having its own history.


For a neutral particle in a uniform medium, the physical cross section is 
independent of $\ell$, and the effective cross section is simply
$\sigma(\ell) \equiv \sigma^{\rm neutral}_\phi$, 
where $\sigma^{\rm neutral}_\phi$ is the macroscopic cross section in this medium.
Its value changes after an elastic interaction, as the projectile energy changed.
The case of a charged particle in a uniform medium is more delicate
as the particle energy evolves during the transport because of energy loss,
or by application of an electric field, for example.
The effective cross-section evolves as
$\sigma(\ell) \equiv \sigma^{\rm charged}_\phi(E(\ell))$,
where $\sigma^{\rm charged}_\phi$ is the macroscopic cross section, the
complication being that $E(\ell)$ is not a unique function but depends on
each particle history.

In a biased scheme, such as the forced collision the effective cross section may
evolve with $\ell$ even for the case of a particle of constant energy.\\

Integrating~(\ref{eqn:bias_rel_loss}), leads to
\begin{eqnarray}
N(\ell) &=& N(0)\cdot \exp\left(-\int_0^\ell {\sigma(s){\rm d}s}\right) \nonumber \\
        &=& N(0)\cdot P_{NI}(0\to\ell),
\end{eqnarray}
where $P_{NI}(0\to\ell) \equiv \exp\left(-\int_0^\ell {\sigma(s){\rm d}s}\right)$
is the non-interaction probability from the origin to distance $\ell$,
% The right term factor $\exp\left(-\int_0^\ell {\sigma(s){\rm d}\ell}\right)$ is
% the fraction of particles free of interactions at distance $\ell$.
% For a single particle, it is the non-interaction probability from the origin to
% distance $\ell$
% \begin{eqnarray}
% \label{PNI_def}
% P_{NI}(0\to\ell) &\equiv& \exp\left(-\int_0^\ell {\sigma(s){\rm d}s}\right),
% \end{eqnarray}
%which is a monotonic decreasing function, as it must be for a non-interaction probability.
and is a monotonic decreasing function, as it must be for a non-interaction probability.
%

%The Markov nature in the differential 
%equation~(\ref{eqn:bias_rel_loss}) is visible in $P_{NI}(0\to\ell)$: if the 
%particle makes an initial step $0\to\ell_1$, followed by a second step
%$\ell_1\to\ell$, with no interaction, $\sigma(\ell)$ is the same function 

The Markov nature of Eq.~(\ref{eqn:bias_rel_loss}) is reflected in 
$P_{NI}(0\to\ell)$: if considering a particle making an initial step $0\to\ell_1$,
followed by a second step $\ell_1\to\ell$, with no interaction,
%
then
$P_{NI}(0\to\ell_1)P_{NI}(\ell_1\to\ell) = P_{NI}(0\to\ell)$,
%
% In an actual Monte Carlo, coordinates for the second step $\ell_1\to\ell$ will
% indeed be changed such that this step will be $0\to\ell-\ell_1$, with the proper
% translation of the cross section function changed:
% $\sigma(s) \to \sigma^\prime(s) \equiv \sigma(\ell_1+s), \forall s \ge 0$.
% For simplicity this shift is not shown in Eq.~(\ref{PNI_def}), which remains 
% numerically the same.
%
% Thus
% \begin{eqnarray}
% \label{PNI_markov}
% P_{NI}(0\to\ell_1)P_{NI}(\ell_1\to\ell) &=& \exp\left(-\int_0^{\ell_1} {\sigma(s){\rm d}s}\right) \exp\left(-\int_{\ell_1}^\ell {\sigma(s){\rm d}s}\right), \nonumber \\
% &=& P_{NI}(0\to\ell),
% \end{eqnarray}
and the probability in $\ell$ does not depend on the previous steps made. This
shows that a biased scheme can still follow a competitive approach between
processes, whether biased and/or non-biased.  The particle flight can be
interrupted at any distance, and non-interaction probabilities in both biased
and analog schemes will be well-defined.

If $p(\ell)$ is the probability density function of interactions, it is related 
to the non-interaction probability by:
\begin{equation}
\label{eqn_PNI_p}
P_{NI}(0\to\ell) = 1 - \int_0^\ell {p(s){{\rm d}s}} ,
\end{equation}
which conversely leads to
\begin{equation}
\label{eqn:p_to_PNI}
p(\ell) = -\frac{{\rm d}}{{\rm d}\ell}P_{NI}(0\to\ell) = \sigma(\ell) P_{NI}(0\to\ell) .
\end{equation}
%Using relation~(\ref{PNI_def}),  equation~(\ref{eqn:p_to_PNI}) can be recast as
%\begin{eqnarray}
%p(\ell) &=& \sigma(\ell) \exp\left(-\int_0^\ell {\sigma(s){\rm d}\ell}\right), \label{eqn:p_to_sigma}
%\end{eqnarray}
%which is also
%\begin{equation}
%\label{eqn:p_to_sigma_PNI}
%p(\ell) =  \sigma(\ell) P_{NI}(\ell).
%\end{equation}
% This has a physical interpretation: 
This shows that the probability 
$p(\ell){\rm d}\ell = P_{NI}((0\to\ell)\cdot \sigma(\ell){\rm d}\ell$
that a particle interacts within the segment ${\rm d}\ell$ at distance $\ell$ is
the product of the probability that it travels $\ell$ without interaction, 
$P_{NI}(0\to\ell)$, and the probability $\sigma(\ell){\rm d}\ell$, that it 
then interacts in the segment ${\rm d}\ell$ \cite{bib:MCNP}.

$\sigma(\ell)$ can also be expressed using $p(\ell)$ and $P_{NI}(0\to\ell)$ 
using
%(\ref{eqn:p_to_sigma_PNI}) and 
(\ref{eqn:p_to_PNI})
\begin{eqnarray}
\label{eqn:sigma_to_PNI}
 \sigma(\ell) &=& -\frac{{\rm d}}{{\rm d}\ell}\log P_{NI}(0\to\ell),
\end{eqnarray}
which is positive, as it must be, as $P_{NI}(0\to\ell)$ is decreasing.
Providing any of the three functions $\sigma(\ell)$, $p(\ell)$ or $P_{NI}(0\to\ell)$
is enough to define the problem.

If several biased independent processes $(i), i=1\dots n$, with effective
cross sections  $\sigma^{(i)}(\ell)$, contribute to particle interactions, each
one is responsible for an interaction amount $-{\rm d}N^{(i)}$ in a segment 
${\rm d}\ell$.  These numbers add up, and Eq.~(\ref{eqn:bias_rel_loss}) 
becomes
\begin{equation}
\label{eqn:bias_rel_loss_many}
\frac{-\sum_i{\rm d}N^{(i)}}{N(\ell)}=\sum_i\sigma^{(i)}(\ell)\cdot {{\rm d}\ell}.
\end{equation}
Equation~(\ref{eqn:p_to_PNI})
%~(\ref{eqn:p_to_sigma_PNI})
keeps the same form, with $\sigma(\ell)$ and
$P_{NI}(0\to\ell)$ becoming the total effective cross section and 
non-interaction probability, given by
\begin{eqnarray}
\sigma(\ell) &=& \sum_i\sigma^{(i)}(\ell), \\
P_{NI}(0\to\ell) &=& \prod_i P^{(i)}_{NI}(0\to\ell).
\end{eqnarray}

A scheme has been implemented in which discrete interaction physics processes
can be biased independently of each other, each possibly having its analog 
interaction law replaced by a biased one. The related analog and biased 
quantities will be denoted by subscripts $a, b$, or by superscripts $(a), (b)$, 
respectively.

In a step, any process which looses the competition for the distance to the next
interaction ends up with a non-interaction.  The biased and analog versions of 
the same process $(i)$ have different probabilities for this to occur, which is
reflected by multiplying the track weight by a non-interaction weight for
each process, and which is
\begin{eqnarray}
\label{eqn:non_int_weight}
w_{NI}^{(i)}(0\to\ell) &\equiv& \frac{P^{(i)}_{NI;\,a}(0\to\ell)}{P^{(i)}_{NI;\,b}(0\to\ell)} .
\end{eqnarray}
For the process which wins the race, the total weight to be applied is
\begin{eqnarray}
w_{\rm total}^{(i)}(\ell) &=& \frac
{p^{(i)}_a(\ell){\rm d}\ell}
{p^{(i)}_b(\ell){\rm d}\ell} , \nonumber \\
&=& w_{NI}^{(i)}(0\to\ell)\cdot w_{I}^{(i)}(\ell) ,
\end{eqnarray}
where $w_{I}^{(i)}(\ell)$ is called the interaction weight, given by
\begin{eqnarray}
\label{eqn:int_weight}
w_{I}^{(i)}(\ell) &\equiv& \frac
{\sigma^{(i)}_a(\ell)}
{\sigma^{(i)}_b(\ell)} .
\end{eqnarray}
Even for this process, it is seen that the non-interaction weight is involved.

In summary, for a track taking a step of length $\ell$, each process $(i)$
multiplies the track weight by its non-interaction weight $w_{NI}^{(i)}(\ell)$~\cite{bib:Weller};
in addition, the process winning the competition for the distance to the next
interaction further multiplies the track weight by its interaction weight 
$w_{I}^{(i)}(\ell)$.\\

The design of occurrence biasing relies on the 
\gclass{G4VBiasing\allowbreak{}Interaction\allowbreak{}Law} class
% §§ ->>>
which defines an abstract interface to model interaction laws.
The $P^{(i)}_{NI}(0\to\ell)$ and $\sigma^{(i)}(\ell)$ calculations are to be
provided by overriding, respectively, the pure virtual methods
% §§ <<<-
\begin{itemize}
\item \gclass{G4double ComputeNonInteraction\allowbreak{}ProbabilityAt(G4double l)} and
\item \gclass{G4double ComputeEffective\allowbreak{}CrossSectionAt(G4double l)},
\end{itemize}
% §§ where \gclass{l} is the length of the step taken.
where \gclass{l} is a re-notation of the length $\ell$ of the step taken.

For a physics process $(i)$ under the control of a 
\gclass{G4\allowbreak{}Biasing\allowbreak{}Process\allowbreak{}Interface} 
instance $(I)$, $(I)$ collects the process cross section at the beginning of
the step and asks the current biasing operator for a potential occurrence 
biasing operation.  If such an operation is provided, it comes with a 
\gclass{G4VBiasing\allowbreak{}Interaction\allowbreak{}Law} that the operator
samples to cause $(I)$ to compete for the next interaction.  The process cross
section is used to build a version of 
\gclass{G4VBiasing\allowbreak{}Interaction\allowbreak{}Law} implementing a 
classical exponential law.  In the subsequent \gclass{Along\allowbreak{}Step}
% §§ ->>>
stage, all instances $(I)$ provide the non-interaction weight
$w_{NI}^{(i)}(0\to\ell)$ (Eq.~(\ref{eqn:non_int_weight})) of their physics process $(i)$
% §§ <<<-
using the method 
\gclass{Compute\allowbreak{}Non\allowbreak{}Interaction\allowbreak{}Probability\allowbreak{}At(l)}
of the biased and analog laws.  In the following \gclass{Post\allowbreak{}Step}
stage, if an instance $(I)$ won the next interaction race, it further applies the weight
for interaction
% §§ ->>>
$w_{I}^{(i)}(\ell)$ (Eq.~(\ref{eqn:int_weight})) of process $(i)$
% §§ <<<-
using the
\gclass{Compute\allowbreak{}Effective\allowbreak{}Cross\linebreak[1]SectionAt(l)} 
method of the two laws.  This interaction weight multiplies the weight of all 
the final state tracks
% §§ ->>>
% §§ produced in
(primary or secondaries) issued from the interaction. This final state may
be the one of the analog versions of the physics process, or a final-state-biased
version of it.\\
% §§ <<<-

In release 10.0, a few concrete occurrence biasing functionalities were provided.
The class \gclass{G4BOptnChangeCrossSection} is a biasing operation to change a
process cross section.  Given this change can be done per process and on a 
step-by-step basis, it can be used to implement schemes similar to the 
exponential transform \cite{bib:exptran}.

The class \gclass{G4BOptrForceCollision} is a biasing operator that reproduces
the forced collision scheme of MCNP \cite{bib:MCNP}.  It handles several 
biasing operations to achieve this: a splitting operation to make a copy of the
track entering the volume, an operation to force the primary track to cross the
volume without interaction (and no tally) and update its weight accordingly, 
and an operation to force the collision on the primary clone, using the total 
interaction cross section collected from the physics processes.

These two schemes have been validated on neutral tracks.

