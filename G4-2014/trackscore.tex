%%%%%%%%%%%%%%%%%%%%%%%%%%%%%%%%%%%%%%%%%%%%%%%%%%%
% trackscore.tex
%%%%%%%%%%%%%%%%%%%%%%%%%%%%%%%%%%%%%%%%%%%%%%%%%%%
\label{sec:trackscore}
\subsection{\textbf{Tracking and Scoring}}

\subsubsection{Design changes in tracking}

The main changes in tracking include easier physics process implementation, new
infrastructure for the re-acceleration of charged particles in electric fields, 
and ``reverse Monte Carlo''.  The problem of re-acceleration is not yet solved 
and requires further development.

Adjoint or ``reverse'' Monte Carlo has been available in \Gfour{} since release 
9.3 and modifications to tracking were required to make this possible.  The 
enabling classes have names beginning with \texttt{G4Adjoint}.  Details of these 
classes and adjoint Monte Carlo can be found in section \ref{sec:biasrev}.

\subsubsection{Concrete scorers}

In \Gfour{}, a hit is a snapshot of a physical interaction or accumulation of 
interactions of tracks in a sensitive detector component.  ``Sensitive'' here
refers to the ability of the component to collect and record some aspects of 
the interactions, and to the \Gfour{} classes which enable this collection.
Because of the wide variety of \Gfour{} applications, only the abstract classes
for both detector sensitivity and hit had thus far been provided in the toolkit.
A user was therefore required to have the expertise necessary to implement the 
details of how hits were defined, collected and stored.

To relieve many users of this burden, concrete primitive scorers of physics 
quantities such as dose and flux have been provided which cover general-use 
simulations.  Flexibly designed base classes allow users to implement their own
primitive scorers for use anywhere a sensitive detector needs to be simulated.

Primitive scorers were built upon three classes, 
\gclass{G4Multi\allowbreak{}Functional\allowbreak{}Detector},
 \gclass{G4VPrimitiveScorer} and
\gclass{G4VSDFilter}.
\gclass{G4\allowbreak{}Multi\allowbreak{}Functional\allowbreak{}Detector}
is a concrete class
derived from \gclass{G4VSensitive\allowbreak{}Detector} and attached to the
detector component.  Primitive scorers were developed on top of the base class
\gclass{G4VPrimitiveScorer}, and as such represent classes to be registered to
the \gclass{G4MultiFunctionalDetector}.  \gclass{G4VSDFilter} is an abstract
class for a track filter to be associated with a
\gclass{G4MultiFunctional\allowbreak{}Detector} or a primitive scorer.
Concrete track filter classes are also provided.  One example is a charged track
filter and a particle filter that accept for scoring only charged tracks and a
given particle species, respectively.

A primitive scorer creates a \texttt{G4THitsMap} object for storing one physics
quantity for an event.  \texttt{G4THitsMap} is a template class for mapping an 
integer key to a pointer value.  Since a physics quantity such as dose is 
generally accumulated in each cell of a detector component during an event or 
run, a primitive scorer generates a \gclass{G4THitsMap<G4double>} object that 
maps a pointer to a \gclass{G4double} for a physics quantity, and uses the cell
number as the integer key.  If a cell has no value, the \gclass{G4THitsMap} 
object has no corresponding entry and the pointer to the physics quantity 
returns a null.  This was done to reduce memory consumption, and to distinguish
an unfilled cell from one that has a value of zero.  The integer key of the cell
is taken from the copy number of the \texttt{G4LogicalVolume} of the detector 
component by default.  \Gfour{} also provides primitive scorers for 
three-dimensional structured geometry, in which copy numbers are taken at each 
of the depth levels at which of the logical volumes are nested in the geometric
structure.  These copy numbers are then serialized into integer keys. 

\subsubsection{Command-based scoring}

Command-based scoring is the easiest way to score primitive physics quantities.
It is based on the primitive scorers and navigation in an auxilliary geometry 
hierarchy (``parallel world,'' Section \ref{sec:nav}).  Users are not 
required to develop code, as interactive commands are provided to set up the 
scoring mesh.

The scoring mesh consists of a scoring volume in a three-dimensional mesh with
a multifunctional detector, primitive scorers and track filters which may be 
attached to a primitive scorer.  An arbitrary number of primitive scorers can 
be registered to the scoring mesh.

A given physics quantity, or score, is accumulated in each cell during a run. 
Interactive commands allow scores to be dumped into a file and written out in 
CSV format.  Other commands allow scores to be visualized as color histograms
drawn on the visualized geometry in either a projection or a selected profile.
 
Because scoring volumes are placed in a parallel world, the scoring volume and
the mass volume are allowed to overlap.  Command-based scoring can therefore 
obtain the physics quantity in an arbitrary volume or surface independent of the
mass volume.  One exception to this is the dose primitive scorer, in which 
the scoring volumes and their cells must coincide exactly with the mass geometry
structure.  This is because the mass density of the cell is taken from the mass 
geometry while the volume of the cell is taken from the scoring geometry. 

%
% Check that mass geometry and scoring geometry are defined elsewhere
%
Most of the command-based scoring is handled in two classes,
\gclass{G4Scoring\allowbreak{}Manager} and \gclass{G4VScoringMesh}.
\gclass{G4ScoringManger} is the administrative class of command-based scoring.
It creates a singleton object which keeps a list of registered scoring meshes,
and operates the scoring according to interactive commands and commands from 
the \Gfour{} kernel. \gclass{G4VScoring\allowbreak{}Mesh} is the base class of 
scoring meshes which represent scoring geometries.  It keeps a list of 
associated primitive scorers.  The \gclass{G4VScoring\allowbreak{}Mesh} object
creates a series of \gclass{G4THitsMap} objects in which each primitive scorer
can accumulate physics quantities in a run. 

Command-based scoring works in both sequential and multithreaded modes.  In
sequential mode, \gclass{G4RunManager} creates scoring meshes at the beginning
of a run.  After each event, the \gclass{G4THitsMap} object in
\gclass{G4VScoring\allowbreak{}Mesh} is updated by adding values in that event
to the primitive scorer.  In multithreaded mode \gclass{G4MTRunManager} creates
a master scoring manager.  Each worker thread of \texttt{G4WorkerRunManager} 
creates its own local scoring manager with scoring meshes.  However, the logical
volume of the scoring mesh is shared between master and local scoring managers.
The local scoring manager then accumulates physics quantities in the same manner
as sequential mode.  At the end of a thread, the worker run manager requests the
master run manager to merge the scores of the local scoring manager with those
of the master scoring manager. 

