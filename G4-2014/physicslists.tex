%%%%%%%%%%%%%%%%%%%%%%%%%%%%%%%%%%%%%%%%%%%%%%%%%%%
% physicslists.tex
%%%%%%%%%%%%%%%%%%%%%%%%%%%%%%%%%%%%%%%%%%%%%%%%%%%
\label{sec:physlists}
\subsection{\textbf{Physics Lists}}
In \Gfour{}, physics lists refer to classes which provide the means to collect
and organize the particle types, physics models and cross sections required for
a particular simulation application.  These classes allow physics processes to 
be registered to the run manager which in turn attaches them to tracks so that 
they may interact properly with the simulation geometry.

When originally conceived, physics lists were intended to give the user maximum 
flexibility and responsibility to choose and combine particles, models and cross
sections.  Developers thus did not provide default physics or specific physics 
combinations to users, except in certain custom situations.  It eventually 
became clear from user requests that ready-made and validated physics modules 
were desired which could be easily plugged into user physics lists.  This led to
the development of several ``reference'' physics lists which were specialized to
provide standard behavior in various application domains.  In medical or space 
applications, for example, detailed atomic physics may be required, while for 
high energy physics it is not.  In high energy applications TeV scale physics is
important, but not for medim and low energies.   

While the basic, maximally flexible physics list classes are still available
and fully documented in the \Gfour{} Application Developers Guide
\cite{bib:AppDevGuide}, the focus here is on prepared, modular physics lists 
which are organized around builders and constructors.  

\subsubsection{Constructors} \label{subsec:ctors}
All prepared physics lists in \Gfour{} derive from the class
\gclass{G4VModular\allowbreak{}PhysicsList} which in turn derives from the base
class \gclass{G4VUserPhysicsList}.  \gclass{G4VModular\allowbreak{}PhysicsList}
maintains a vector of physics modules, each of which is an implementation of the 
\gclass{G4VPhysics\allowbreak{}Constructor} class.  A module, referred to here 
as a physics constructor, enables the logical grouping of related physics 
models, cross 
sections and particles.  This allows the most accurate and CPU-appropriate 
physics models to be applied to given energy ranges and particle types.  The 
\gmethod{ConstructParticle()} and \gmethod{ConstructProcess()} methods of 
\gclass{G4VPhysics\allowbreak{}Constructor} can be used to instantiate the 
particle types needed for a given application, and to assign models and cross
sections to 
processes.  For example, all pions and kaons could be instantiated in a meson 
physics constructor, which would also assign two hadronic models, a cascade 
model at low energies and parton string model at high energies, to the 
processes pertaining to pions and kaons.

The chosen electromagnetic and hadronic constructors are stored in a vector, 
which makes it easy to build a new physics list by adding, removing or replacing
physics constructors.  A large collection of these constructors is included in 
the \Gfour{} toolkit.  The electromagnetic constructors are listed in 
Table~\ref{em:physl} and examples for using them \cite{bib:uni} are distributed
with the release. 

\begin{table*}
\caption{List of default and optional \Gfour{} EM physics constructor classes.
         One of several optional EM physics constructors may be chosen by 
         appending its shorthand name, listed in the ``Extension'' column, to 
         the name of a basic physics list, such as FTFP\_BERT\_ENV, for example.
         WVI refers to the Wenzel multiple scattering model as implemented by 
         V. Ivantchenko.}
\label{em:physl}
\begin{center}
\begin{tabular}{llll}
\hline
Physics Constructor Name& Application& Extension & Comment \\ \hline
\gclass{G4EmStandardPhysics} & HEP &  &default (ATLAS)\\
\gclass{G4EmStandardPhysics\_option1} & & EMV & simplified (CMS)\\
\gclass{G4EmStandardPhysics\_option2} & & EMX & simplified (LHCb)\\
\hline
\gclass{G4EmStandardPhysics\_option3} & space \& & EMY & detailed  \\
                             &  medicine &     & standard models\\
\gclass{G4EmLivermorePhysics} & & LIV & detailed  \\
                     &       &          &  Livermore models\\
\gclass{G4EmPenelopePhysics} &  & PEN & detailed \\ 
                    &  &     & Penelope models\\
\gclass{G4EmStandardPhysics\_option4} &  & EMZ & combining \\
                             &  &     & best models\\
\hline
\gclass{G4EmLivermorePolarizedPhysics} &  &  & polarized models\\
\gclass{G4EmLowEPPhysics} &  &  & new low energy models\\
\gclass{G4EmStandardPhysicsWVI} &  &  & WVI multiple scattering\\
\gclass{G4EmStandardPhysicsSS} &  &  & single scattering \\
\hline
\gclass{G4EmDNAPhysics} & DNA &  & default for DNA physics \\
\gclass{G4EmDNAPhysics\_option1} & DNA &  & WVI multiple scattering \\
\hline
\gclass{G4OpticalPhysics} & all &  & production and transport \\
                 &     &  & of optical photons \\
\hline
\end{tabular}
\end{center}
\end{table*}

\subsubsection{Builders}
It is convenient to implement the physics constructors with more granular physics
components.  As an example, consider the physics constructor 
\gclass{G4HadronPhysicsFTFP\_BERT}, which implements all inelastic hadronic 
interactions by using the FTF parton string and Bertini cascade models.  It 
implements the \gmethod{G4HadronPhysicsFTFP\_BERT::ConstructProcess()} method by
instantiating and invoking several builder classes, such as 
\gclass{G4FTFP\allowbreak{}Neutron\allowbreak{}Builder},
\gclass{ G4BertiniPiKBuilder} and \gclass{G4HyperonFTFPBuilder}.  

Each type of builder has its own class which assigns physics models and cross 
sections to processes.  It is here where the overlap in energy ranges between 
two models is decided.  For an energy $E$ in the overlap region $E_1 < E < E_2$,
one of the two models is chosen randomly; the probability to choose the model 
valid at higher energy is zero at $E_1$ and one at $E_2$, increasing linearly 
with energy.  It is also here where models are built from sub-models.  More 
complicated generators, like the FTF parton string model or nuclear 
de-excitation, are not implemented as a single model, but as a collection of 
them.  This level of detail is not usually of interest to users, hence its 
encapsulation within the builder classes.

\subsubsection{Reference Physics Lists}
As of release 10.0 the toolkit provided nine reference physics lists whose names
reflect the combination of models used to describe the hadronic interactions 
necessary for various applications.  Unless otherwise indicated, the 
electromagnetic interactions in these physics lists are described by the 
\Gfour{} standard EM models.  Reference physics lists are extensively and 
routinely validated.    

Perhaps the most-used reference physics list for high energy and space 
applications is FTFP\_BERT.  It uses the \Gfour{} Bertini cascade for 
hadron-nucleus interactions from 0 to 5 GeV incident hadron energy, and the FTF
parton string model for hadron-nucleus interactions from 4 GeV upwards.  The 
letter $P$ indicates that the \Gfour{} precompound mode is used to de-excite 
the nucleus after the high energy FTF interaction has been completed.  The 
FTFP-Bertini combination forms the backbone of many physics lists. 

QGSP\_BERT is a popular alternative which replaces the FTF model with the 
QGS model over the high energy range.  Using the other intranuclear cascade 
models, \gclass{G4BinaryCascade} or \gclass{G4INCLXX} instead of Bertini, 
produces the FTFP\_BIC, QGSP\_INCLXX and QGSP\_BIC physics lists, of which the
latter is often used in medical applications.  When high precision neutron 
propagation is included in a physics list, the letters $HP$ are appended to its 
name, for example FTFP\_BERT\_HP.  Many other physics lists can be built in 
this way, but only a handful of them are sufficiently tested and validated to
qualify as reference lists.

There are also specialized physics lists such as QBBC \cite{bib:QBBC} and 
Shielding, which are not currently considered to be reference lists, but are 
often used.

As mentioned in sections \ref{sec:emphys} and \ref{subsec:ctors}, there are 
several collections of electromagnetic models besides the standard.  Using these
in a physics list is made easy by the \gclass{G4PhysicsListFactory}.  A user 
need only specify the desired electromagnetic option, and the factory will 
substitute it for the standard collection in the newly created physics list.  
Examples of physics lists that may be created using G4PhysicsListFactory are:
\begin{itemize}
\item QGSP\_BERT\_EMV, the QGSP\_BERT set of hadronic models with faster but
      less precise electromagnetic models,
\item FTFP\_BERT\_LIV, the FTFP\_BERT set of hadronic models with the Livermore
      electromagnetic models, or
\item QGSP\_BIC\_DNA, the QGSP\_BIC set of hadronic models with the low energy
      DNA electromagnetic models.
\end{itemize}

