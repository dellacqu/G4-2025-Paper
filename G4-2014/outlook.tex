%%%%%%%%%%%%%%%%%%%%%%%%%%%%%%%%%%%%%%%%%%%%%%%%%%%
% outlook.tex
%%%%%%%%%%%%%%%%%%%%%%%%%%%%%%%%%%%%%%%%%%%%%%%%%%%
\label{sec:outlook}
\subsection{\textbf{A Brief Summary of \Gfour{} Progress}}
Major changes and developments in the \Gfour{} toolkit took place between the
8.1 and 10.1 releases.  These include:

\begin{itemize} 
  \item the migration to multithreading, 
%  \item the switch from gmake to CMake,
  \item the addition of tessellated solids and a unified geometry description,
  \item general biasing methods,
  \item improved and expanded physics models, 
  \item expanded validation and testing, 
  \item reference physics lists,
  \item the addition of rudimentry analysis methods, and
  \item improved and expanded visualization tools.
\end{itemize}

As a result the toolkit is more versatile, easier to use and makes more 
efficient use of available CPU and memory resources.

\subsection{\textbf{New Directions}}
These changes were made in response to the demands of a diversifying user 
community and to the opportunities made available by advancing technology.  It
is expected that both these trends will continue and that \Gfour{} will 
continue to evolve with them.

With this in mind the \Gfour{} collaboration is studying new options for the 
future.  GPUs and accelerated processors offer great potential for speeding up 
computationally intensive applications, and could possibly be adapted for use in
physics simulations.  Massively parallel computing and vectorization are also 
being examined as a way to exploit available supercomputer capacity
\cite{bib:Dotti,bib:Szo}.

The drive to make \Gfour{} easier to use will continue.  An increasing percentage
of the \Gfour{} user base requires more turn-key operation and better 
documentation.  To enable this, improved user interfaces, simplified physics  
choices and still more powerful build tools will be required.

The expansion of \Gfour{} into new physics domains will also continue. 
Users in nuclear physics require more detailed nuclear reactions and models, 
space and medical applications depend increasingly on precise, fast 
electromagnetic and radioactive decay modeling, biophysics and material 
science continue to expand the simulation of chemical kinetics and damage to 
micro-structures, and photon science is expected to be a user of radiation 
damage and accelerator dark current simulations.  While new capabilities are 
currently being developed to meet the needs of experiments at the high energy,
intensity and cosmic frontiers, it is clear that the increasing use of \Gfour{}
in other areas will also lead to new toolkit developments.

