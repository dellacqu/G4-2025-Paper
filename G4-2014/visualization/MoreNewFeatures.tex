
% NewFeatures.tex
Time Slicing was added to allow one to produce movies that show the time
development of an event.  With time slicing enabled, the OpenGL line segments
that represent a particle trajectory are annotated with time information.  Users
can then select an OpenGL view that corresponds to a given time, and a sequence
of such views produces the frames of a time development movie.  Users can 
produce these movies in any OpenGL viewer by the appropriate use of \Gfour{}
command macros.  The Qt driver provides a simplified way for users to make such
movies.

\Gfour{} visualization now has the ability to retain the pointers to 
previously-viewed events, so that after visualizing a set of events, one can go
back to the beginning of the set and review the events.  When coupled with 
customized user code that specifies which events should be kept, one can 
potentially run a very large set of events and then afterwards choose to 
visualize only those events that satisfied some personal set of trigger 
conditions.

The following features have also been added:
\begin{itemize}
\item parallel worlds, including layered mass worlds, may now be viewed
      individually or superimposed on the main geometry world;

\item magnetic fields may be displayed as a set of arrows indicating local
      field direction, with arrow lengths proportional to field strength;

\item decorations are now supported which allow the user to easily annotate
      graphical views with text (placed either in 3D coordinates or 
      in the 2D coordinates of the graphics window), run and event number,
      arrows, axes, rulers, date stamps and logos;

\item users may adjust the visibility or appearance of geometry by using the 
      /vis/geometry commands which globally modify the appearance of some set of
      geometry objects, while the /vis/touchable commands allow control of these
      objects individually.
\end{itemize}

