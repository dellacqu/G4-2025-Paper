
Many options are now provided for how trajectories should be modeled (how colors
or line styles are selected).  These improvements have eliminated the most 
common reason users had to code their own trajectory classes.  In addition to 
the default model, where trajectories were colored by charge, one can now set 
color or other line properties based on particle ID, particle origin volume, or
any other particle attribute that has been loaded into a \texttt{G4AttValue}.
One can also add entirely new, customized trajectory models.  New options make
it easy to control whether trajectories are shown as basic lines, lines plus
step points or step points alone, and one may also modify step point colors.

Additional new features allow trajectories to be filtered, causing only a 
specific subset to be drawn.  These filtering options match the design of the 
trajectory modeling options, so that filtering based on charge, particle ID, 
particle origin volume, or some custom aspect, is possible.  Filters may be
daisy-chained so that one may show, for example, only the neutrons originating
from a particular collimator.

Completing the set of additions to trajectory drawing is the ability to select
smooth and rich trajectories.  By default, trajectories are represented as a set
of line segments connecting particle steps.  Because \Gfour{}'s efficient 
stepping algorithm may require very few steps in some magnetic fields, the 
default trajectory drawn through a solenoidal field may appear very jagged.  The
optional Smooth Trajectory Drawing causes additional points to be generated
along the particle trajectory so that the visualization is smoother.  Rich 
trajectories concern the amount of additional information with which 
trajectories and step points are annotated.  By default, trajectories have only 
basic information attached and step points have only position information; thus 
when one picks on these objects in the various pick-enabled viewing systems 
(HepRApp, Qt, OI or OpenGL with X11), one discovers only a few pieces of 
information about the trajectory and no details about the trajectory points.  
The Rich trajectory option enriches this annotation, providing picked 
trajectories containing many more pieces of information, such as the entire 
history of geometry volumes traversed.  It also adds a wealth of information to
step points, such as the type of process that created the step.

