
The final set of changes concern \Gfour{}'s migration to multithreaded (MT) 
operation.  The overall design of visualization remains little-changed for 
those users running in sequential mode, but significant changes were required 
to enable visualization from MT mode.

Currently in MT mode, events are only drawn at end of run, that is, once all
threads have completed their work.  This limitation is scheduled to be removed
in release 10.2 by moving part of visualization to its own thread, such that 
each event is available for drawing as soon as that event is complete.

In MT mode, visualization will properly handle any commands that request drawing
of high level graphical objects (geometry volumes, trajectories and decorations
such as axes).  However, user-supplied code that directly asks the visualization 
system to draw low level graphical primitives (polygons or polylines) is not 
supported.  This limitation will not likely affect many \Gfour{} users, as 
recent improvements to geometry, trajectory and decoration handling have made 
such user-supplied code largely unnecessary.  Because significant work will be 
required to remove this limitation, support will come only if there is strong 
demand for these features in MT mode.

The RayTracer driver has itself been been multithreaded to take maximum 
advantage of MT.

