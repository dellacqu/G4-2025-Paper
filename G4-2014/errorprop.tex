%%%%%%%%%%%%%%%%%%%%%%%%%%%%%%%%%%%%%%%%%%%%%%%%%%%
% errorprop.tex
%%%%%%%%%%%%%%%%%%%%%%%%%%%%%%%%%%%%%%%%%%%%%%%%%%%
\label{sec:errorprop}

\subsection{\textbf{Error Propagation}}

The track error propagation package serves to propagate one particle together 
with its error from a given trajectory state (i.e. position and momentum with
errors) until a user-defined target is reached (a surface, volume or given
track length).  Its main use is for the track fitting of simulated or real data
to reconstruct the trajectory of a particle that has left several detector
signals along its path. 

To do its work, this package uses \Gfour{} geometry and physics, so that the 
same geometry and magnetic field used in the simulation can be used for the 
track error propagation.  The \Gfour{} physics equations can also be used, but
it is the average trajectory that should be propagated and this must be taken 
into account.  Although the user may use his/her own physics list, a physics 
list provided in the package is recommended.  It has no straggling due to 
multiple scattering, no fluctuations in the energy loss, no emission of 
secondary particles and no hadronic processes.  This physics list also 
accommodates backward propagation as well as forward (the direction the track 
followed to produce the detector signals).

When a track is propagated forward, it loses energy and the energy at the 
beginning of the step is used to calculate the energy loss.  When a track is
propagated backward, it gains energy and the energy at the end of the step is
used.  Thus, depending on propagation direction, quite different values of the
energy loss might be obtained.  To avoid this, the algorithm uses in both cases
the average energy to compute the energy loss.

Propagation is terminated when one of the following targets are met:
\begin{itemize}
\item \emph{Surface}: a user-defined surface is reached.  The surface does not
      have to be part of the geometry, as the propagation takes into account 
      both the geometry and the user-defined surfaces at the same time.  
      Currently plane and cylindrical surfaces may be defined, as well as
      surfaces formed from the combination of the two;
\item \emph{Volume}: the surface of one of the logical volumes of the geometry
      is reached.  The user may choose if the track is to stop only when 
      entering the volume, exiting the volume, or in both cases.
\item \emph{Track length}: a given track length is reached.
\end{itemize}

This package was implemented following GEANE~\cite{errprop:geane} from GEANT3,
which was based on the equations developed by the EMC 
collaboration~\cite{errprop:emc}. 

Users may implement the example provided with the \Gfour{} distribution, at 
\verb"examples/extended/errorpropagation".  The geometry in this example 
simulates a simplified typical high energy physics detector consisting of
\begin{itemize}
\item  an air-filled beamline,
\item  an air-filled central detector, 
\item  a copper calorimeter, divided into four sections,
\item  an aluminium calorimeter, divided into ten sections, and
\item  an air-filled muon detector.
\end{itemize}
While the example does not pretend to have a fully realistic geometry, it is 
sufficient to test the code.  Also the volumes were chosen to match those of
the example in the GEANE paper so that a detailed comparsion of results could
be done.
  
The detector is immersed in a magnetic field with a default value of 1 kilogauss
pointing along the negative $z$ axis.  This value can be changed by user command.
An initially free trajectory state is created to simulate a muon track of 20 GeV
along the $x$ axis.  This track is propagated until one of the termination 
targets mentioned above is reached.  An enviroment variable serves to select the
type of target.  Another enviroment variable allows either forward or backward 
propagation to be selected.  The user also has the freedom to choose whether
control will be returned to the user after each step, or only after the 
propagation target has been reached.

