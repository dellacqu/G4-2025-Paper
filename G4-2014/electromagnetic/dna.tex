%%%%%%%%%%%%%%%%%%%%%%%%%%%%%%%%%%%%%%%%%%%%%%%%%%%
% dna.tex
%%%%%%%%%%%%%%%%%%%%%%%%%%%%%%%%%%%%%%%%%%%%%%%%%%%
\Gfour{} offers a set of physics processes and models \cite{embib:dna0} to 
simulate track structure in liquid water, the main component of biological
media.  These were developed as part of the \Gfour{}-DNA project 
\cite{embib:dnaweb}, and extend  
% in 2001 by the European 
% Space Agency 
\Gfour{} to include the simulation of biological damage by ionizing radiation
\cite{embib:dna1,embib:dna2}. 

The first set of discrete processes was delivered in 2007 \cite{embib:dnaProc1}.
Their accuracy was further evaluated and improved 
\cite{embib:dnaxs, embib:dnaProc2, embib:dnaProc3} through the inclusion, 
for example, of more accurate modeling of electron elastic scattering
\cite{embib:dnaElast}, and additional physical processes for sub-excitation
electrons, such as vibrational excitation and molecular attachment 
\cite{embib:dnaTS}.  These processes are critical for the modeling of
physico-chemical processes in liquid water \cite{embib:chem:paper1}. 

A major design upgrade of the software classes was applied in order to allow 
their combination with other \Gfour{} EM processes and models, for a coherent 
modeling of EM interactions \cite{bib:uni,embib:dnaPhysList}.  Thus, for 
their simulation applications, users may instantiate a 
\gclass{G4EmDNA\allowbreak{}Physics} object from their physics list.  This 
physics constructor contains all required \Gfour{}-DNA physics processes and 
models for
\begin{itemize} 
\item electrons, including ionization, excitation, elastic scattering, 
      vibrational excitation and attachment,
\item protons and neutral hydrogen, including excitation, ionization, electron
      capture and stripping,
\item alpha particles and their charged states, including excitation,
      ionization, electron capture and stripping, and  
\item ionization for Li, Be, B, C, N, O, Si and Fe ions.
\end{itemize}
 
Stopping powers and ranges simulated with \Gfour{}-DNA have been compared to
international recommendations \cite{embib:dnaStop}.  These processes can be
combined with \Gfour{} gamma processes.  Note that the \emph{Livermore} low
energy electromagnetic gamma models are selected by default in the 
\gclass{G4EmDNA\allowbreak{}Physics} constructor. 

As an illustration, \\
\verb"examples/extended/medical/dna/dnaphysics"\\
explains how to use this physics constructor.  In addition,
\verb"examples/extended/medical/dna/microdosimetry" describes how to 
combine \Gfour{}-DNA and \Gfour{} standard electromagnetic processes in a simple
user application.  A variety of applications based on \Gfour{}-DNA processes and
models allow the study of elementary energy deposition patterns at the 
sub-cellular scale.  For example, the comparison of dose point kernels 
\cite{embib:dnaPK}, $S$-values  \cite{embib:dnaS}, radial doses 
\cite{embib:dnaRad}, cluster patterns for ions with the same LET 
\cite{embib:dnaLET}, the effect of a magnetic field on electron track structures
\cite{embib:dnaMag}, and the modeling of direct DNA damage 
\cite{embib:dnaDirD1,embib:dnaDirD2,embib:dnaDirD3,embib:dnaDirD4}
have so far been explored utilizing these tools.  They even provide a framework
for the future study of non-targeted biological effects \cite{embib:dnaBio}, 
extending further the first applications of \Gfour{} electromagnetic physics at 
the physics-biology frontier
\cite{embib:dnaBio1,embib:dnaBio2,embib:dnaBio3,embib:dnaBio4,embib:dnaBio6,embib:dnaBio7,embib:dnaBio8}.

