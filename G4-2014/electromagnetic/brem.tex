%%%%%%%%%%%%%%%%%%%%%%%%%%%%%%%%%%%%%%%%%%%%%%%%%%%
% brem.tex
%%%%%%%%%%%%%%%%%%%%%%%%%%%%%%%%%%%%%%%%%%%%%%%%%%%
A variety of models to simulate the radiation loss of charged particles are
available in the toolkit (Table~\ref{em:brem}).  Significant efforts were made
\cite{embib:chep12} to improve the description of EM shower shapes in order to
simulate accurate $H\to\gamma\gamma$ signals in the LHC detectors 
\cite{embib:Higgs1,embib:Higgs2}.  High energy EM shower profiles are sensitive 
to electron/positron bremsstrahlung spectra and angular distributions.
All \Gfour{} models of bremsstrahlung in the intermediate energy range 1~keV to
1~GeV are based on tables of differential cross sections published by Seltzer
and Berger \cite{embib:SelzBer}.  Evaluated 2-D tables are stored in the EM 
data set \gclass{G4LEDATA} and are loaded at initialization time.  The 
Ter-Mikaelian suppression of low energy gamma emission due to finite formation 
length (see \cite{embib:bremfl} and references therein) is taken into account 
by all models. 

\begin{table*}
\caption{List of \Gfour{} models for simulation of radiation loss with 
         recommended energy ranges. Array size refers to the internal table
         storing number of primary energy points versus number of secondary 
         energy points.}
\label{em:brem}
\begin{center}
\begin{tabular}{llll}
\hline
Particle& Model& Energy range&Array size \\ \hline
 e-/e+& \gclass{G4SeltzerBergerModel} \cite{embib:chep12}& 1 keV - 10 GeV & 57x32\\
 e-/e+ & \gclass{G4PenelopeBremsstrahlungModel} & 1 keV - 10 GeV & 57x32\\
e-  & \gclass{G4LivermoreBremsstrahlungModel} & 1 keV - 10 GeV & 31x14 \\
 e-/e+ & \gclass{G4eBremsstrahlungRelModel} \cite{embib:chep12}&   1 GeV - 10 PeV &  \\
$\mu^{\pm}$    & \gclass{G4MuBremsstrahlungModel} \cite{embib:emmu}& 1 GeV - 10 PeV & \\
$\mu^{\pm}$    & \gclass{G4MuPairProductionModel} \cite{embib:emmu}& 1 GeV - 10 PeV & 17x1000\\
$\pi^{\pm}, K^{\pm}, p$ &\gclass{G4hBremsstrahlungModel} \cite{embib:hb}&5 GeV - 10 PeV &\\
$\pi^{\pm}, K^{\pm}, p$ &\gclass{G4hPairProductionModel} \cite{embib:hb}&5 GeV - 10 PeV &13x1000 \\
\hline
\end{tabular}
\end{center}
\end{table*}
For $e^{\pm}$ above 1~GeV, a relativistic model \cite{embib:chep12} was developed
with an improved treatment of the LPM effect \cite{embib:migdal}.  This was 
implemented on top of the classical Bethe-Heitler cross section with complete 
screening.  Two types of saturation effects, LPM and formation length, have been 
combined to limit the number of low energy photons produced.  These corrections
have a distinct impact on EM shower shape and fluctuations of energy loss for
high energy EM particles, of particular importance in LHC experiments.

Because simulation of radiation losses of muons is also important for LHC 
experiments, muon bremsstrahlung and pair production models were developed
\cite{embib:emmu}.  The effect of catastrophic energy loss by high energy muon 
bremsstrahlung is well reproduced by simulation and is essential for muon 
identification.  The process of $e^+e^-$ pair production by muons dominates the
average energy loss at high energy \cite{embib:emmu}; proper simulation of the 
final state requires keeping a detailed 2-D internal table of differential cross
sections (Table~\ref{em:brem}) with a structure chosen to achieve a compromise 
between memory usage, initialization time, and accuracy \cite{embib:chep14}.  
Analysis of CMS test beam data \cite{embib:cmstb} indicates that bremsstrahlung
and pair production by pions and protons should be taken into account.  This was
achieved on top of the muon processes by changing the spin term and the mass of 
projectile particles \cite{embib:hb}. 
