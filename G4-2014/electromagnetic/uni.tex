%%%%%%%%%%%%%%%%%%%%%%%%%%%%%%%%%%%%%%%%%%%%%%%%%%%
% uni.tex
%%%%%%%%%%%%%%%%%%%%%%%%%%%%%%%%%%%%%%%%%%%%%%%%%%%
In the early stages of \Gfour{}, low and high energy electromagnetic processes
were developed independently, with the result that these processes could not
be used in the same run.  To resolve this problem, the interfaces were unified
so that the standard, muon, high energy, low energy and DNA EM physics 
sub-packages \cite{bib:uni} now follow the same design.

All \Gfour{} physical processes, including transportation, decay, EM, hadronic,
optical and others, were implemented via the unique general interface
\gclass{G4VProcess}.  Three EM process interfaces inherit from it via the 
intermediate classes \gclass{G4VContinuousDiscreteProcess} or 
\gclass{G4VDiscreteProcess} \cite{embib:design}:
\begin{itemize}
\item \gclass{G4VEnergyLossProcess}, which is active along the step and post
              step,
\item \gclass{G4VMultipleScattering}, which is active along the step and
\item \gclass{G4VEmProcess}, which has no energy loss and is active post step
              and at rest. 
\end{itemize}
These three base classes are responsible for interfacing to the \Gfour{} kernel, 
initializing the electromagnetic physics, managing the energy loss, range and 
cross sections tables, managing the electromagnetic models, and the built-in 
biasing options.  Each process inherits from one of these base classes, and has
one or more physics models.  EM physics models were implemented via the 
\gclass{G4VEmModel} interface.  A model is applied for a defined energy range 
and \gclass{G4Region}, allowing, for example, one model from the low energy and 
one from the high energy sub-package to be assigned to a process for a given 
particle type.

Migration to this common design resulted in an improvement of overall CPU 
performance, and made it possible to provide several helper classes which are 
useful for a variety of user applications:
\begin{itemize}
\item \gclass{G4EmCalculator}: accesses or computes cross section, energy loss,
              and range;
\item \gclass{G4EmConfigurator}: adds extra physics models per particle type, 
              energy, and geometry region; 
\item \gclass{G4EmSaturation}: adds Birks saturation of visible energy in
              sensitive detectors;
\item \gclass{G4ElectronIonPair}: samples ionization clusters in tracking
              devices.
\end{itemize}

These common interfaces enabled the full migration of EM classes to
multithreading \cite{embib:chep14} with only minor modifications of the existing
physics model codes.  Initialization of the energy loss, stopping power and 
cross section tables is carried out only once in the master thread at the
beginning of simulation and these tables are shared between threads at run time. 

Further improvements were made through the factorization of secondary energy and
angle sampling.  The \gclass{G4VEmAngularDistribution} common interface allows
the reuse of angular generator code by models in all EM sub-packages.  The
implementation of a unified interface for atomic deexcitation, 
\gclass{G4VAtomDeexcitation} provides the possibility of sampling atomic 
deexcitation by models from all EM sub-packages.
 
The consolidation of the EM sub-packages boosts the development of new models,
% and automatically resolves a number of problems seen in previous versions.
% Moreover, it 
provides new opportunities for the simulation of complex high energy and low 
energy effects and enables better validation of EM physics \cite{embib:uni2}. 

