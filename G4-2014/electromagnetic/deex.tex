%%%%%%%%%%%%%%%%%%%%%%%%%%%%%%%%%%%%%%%%%%%%%%%%%%%
% deex.tex
%%%%%%%%%%%%%%%%%%%%%%%%%%%%%%%%%%%%%%%%%%%%%%%%%%%

Atomic de-excitation can be activated in all EM physics lists through the common
atomic de-excitation interface \gclass{G4VAtomDeexcitation} \cite{bib:uni}.
Photo-electric effect, Compton scattering, and discrete ionization models 
provide cross sections of ionization for each atomic shell.  The de-excitation
code is responsible for sampling the electromagnetic cascade with fluorescence
and Auger electron emission, and was implemented using evaluated data 
\cite{embib:eadl}. Recently, alternative, more accurate transition energies have
become available in \Gfour{} 10.1 through the addition of a new data set 
\cite{embib:SPaltani}.

The ionization cross section model for Particle Induced X-ray Emission (PIXE) is
based on the condensed history approach.  Specific cross sections can be defined
for electrons, protons, and ions.  Users can select from different sets of 
theoretical or empirical shell ionization cross sections \cite{embib:pixe}.

The simulation of K, L, and M X-ray yields demands knowledge of the X-ray 
production cross sections.  These were calculated using the ECPSSR theory, 
initially developed by Brandt and Lapicki \cite{embib:deex1} and recently 
reviewed \cite{embib:deex2,embib:deex3}.  Computing the X-ray production cross 
sections from first principles is a time-consuming process due to the numerical 
double integral of form factor functions needed to obtain the ionization cross 
sections for each shell or sub-shell (Eq.(23) of \cite{embib:deex2}), over all 
possible values of the energy and momentum transfer of the incident particle.  

The calculation was expedited through the use either of extensive tables and
interpolation methods, or efficient algorithms providing sufficiently good
approximations.
% Aside from empirical tabulations data and a full analytical
% calculations method, efficient algorithms were implemented to determine K, L 
% and M-shells ionization cross-sections
% for H and He ions. 
Algorithms were implemented based on the ECPSSR ionization cross sections for
H and He ions calculated for the K and L shells using the form factor functions
for elements with atomic number 6 to 92 over the energy range of 0.1 to 100 
MeV.  In the case of the M shells, the ionization cross sections are given 
for elements with atomic number 62 to 92 over the energy range of 0.1 to 10 MeV.
Furthermore, the tables generated to develop the algorithms were obtained by
the integration of the form factor functions that describe the process using 
Lobatto's rule \cite{embib:deex4}, and are also available.  The cross sections
generated by the algorithms deviate less than 3\% from the tabulated values, 
roughly matching the scatter of empirical data \cite{embib:deex2}.  Further 
details and considerations of these calculations can be found in 
\cite{embib:deex2,embib:deex3}.  Comparisons of simulated and experimental 
spectra obtained under proton irradiation of several materials are shown in
\cite{embib:deex5,embib:deex6}.

