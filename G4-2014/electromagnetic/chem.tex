%%%%%%%%%%%%%%%%%%%%%%%%%%%%%%%%%%%%%%%%%%%%%%%%%%%
% chem.tex
%%%%%%%%%%%%%%%%%%%%%%%%%%%%%%%%%%%%%%%%%%%%%%%%%%%

Radiation chemistry is the science of the chemical effects of radiation on
matter.  Simulation tools in this field are relevant to many applications, such
as the production of polymers by the irradiation of monomers.  However, one of
the most studied materials under irradiation is liquid water, because it is used
as a coolant in nuclear power plants and its irradiation may create oxidative 
species that are likely to initiate the corrosion process.  Water is also of 
interest because it is the main component of biological materials.

When biological targets are exposed to radiation, the chemical response can be 
complex, depending on the composition of the medium as well as on the energy and
type of radiation.  For example, water radiolysis (dissociation of water by 
radiation) promotes the creation of oxidative species.  These species can either
react with one another or with the biological materials, interfering with the 
normal functioning of one or many cells.

In the context of the \Gfour{}-DNA project, a prototype for simulating radiation 
chemistry was developed \cite{embib:dna3, embib:chem:paper1} and delivered with
\Gfour{} version 10.1.  It is now possible to simulate the physical stage, the 
physico-chemical stage (lasting up to about 1 picosecond) and the chemical stage
(from 1 picosecond up to 1 microsecond) of water radiolysis.
%   The simulation proceeds in two stages, the first 
% consisting of prompt physics processes (reaction times $ < 10^{-9} s$) which 
% provide a set of wounded biological and water molecules in a volume of interest, 
% and the second consisting of physico-chemical processes with reaction times up
% to $10^{-6} s$. 

The \Gfour{}-DNA physical processes may in some cases create
water molecules which are electronically modified, that is, they may be ionized,
excited or have extra electrons in the case of dissociative attachment.  The 
electronic and atomic readjustment of the water molecules can eventually lead to
their dissociation.  The dissociation of water molecules is taken into account 
by random selection using user-defined branching ratios \cite{embib:chem:paper1}.
The positioning of the dissociative products is defined by the 
\gclass{G4DNAWater\allowbreak{}DissociationDisplacer} class from qualitative 
considerations \cite{embib:chem:Kreipl2009}.  It is assumed that the 
dissociation of water molecules is independent of the physical stage of the 
simulation.  This is to say that only electronic modifications undergone by the
water molecule are responsible for the dissociative pathways.  The branching 
ratios and positioning of dissociative products may be adjusted by the user if 
necessary.

Dissociative products can recombine to form new chemical species. To take this 
recombination into account, a stepping algorithm was developed specifically for 
managing collisions between \Gfour{} tracks.  The purpose of this algorithm is 
to synchronize the transport of tracks during simulation.  This means that all 
tracks are transported over the same time, accounting for chemical reactions 
that may happen during a given time step. A description of this synchronized 
stepping, applied to radiation chemistry, is available 
in \cite{embib:dna3,embib:chem:these}.

This stepping mechanism could, in principle, be applied to applications other
than radiation chemistry.  A process must first be made compatible with the 
\gclass{G4VITProcess} interface, which contains track information.  To run a 
radio-chemistry simulation, a table of reactions describing the supported 
reactions and the corresponding reaction rates must be provided.

To simplify the use of the chemistry module, the 
\gclass{G4EmDNA\allowbreak{}Chemistry} constructor provides default settings,
and three examples, 
\verb"examples/extended/medical/dna/chem1", 
\verb"examples/extended/medical/dna/chem2" and
\verb"examples/extended/medical/dna/chem3", are included.  These examples 
progressively demonstrate how to activate and use the chemistry module from a
user application. The module may be used with or without irradiation.

The chemistry module is compatible with the current event-level multithreaded 
mode of \Gfour{}.  However, the use of the module in this mode with a large number
of simulated objects or threads is not recommended because synchronized stepping
requires intense memory usage.

For now, the chemistry module allows prediction of the chemical kinetics induced 
by ionizing radiation. A future goal of the \Gfour{}-DNA project is to account for
DNA damage resulting from irradiation. 
