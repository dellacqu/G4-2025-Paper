%%%%%%%%%%%%%%%%%%%%%%%%%%%%%%%%%%%%%%%%%%%%%%%%%%%
% xray_opt.tex
%%%%%%%%%%%%%%%%%%%%%%%%%%%%%%%%%%%%%%%%%%%%%%%%%%%
\Gfour{} can accurately simulate nonlinear scintillators where the light yield is
a function of particle type, energy deposition and kinetic energy of the 
ionizing particle \cite{em:opt1}.  In scintillators with a linear response, 
light production is directly proportional to the ionizing energy deposited and
the total light produced can be computed as the sum of the light produced during
small simulation steps without regard for the kinetic energy of the ionizing 
particle at each energy-depositing step.

In scintillators with a nonlinear response, the yield during a step is
calculated as the difference in yields for hypothetical, totally absorbed 
particles at the kinetic energies before and after the step.  This method 
correctly models the total light produced by a multiple step ionizing particle 
track, and accounts for two important cases.  In the first case, light is 
produced correctly for incomplete energy deposition of a charged particle, such
as when the particle exits the scintillator volume or when the particle is 
absorbed in a nuclear reaction.  In the second case, the scintillation photon 
density is larger in the high kinetic energy portion of the track for the usual
case where the nonlinear photon yield increases with particle energy.  This 
enables the precision simulation of organic or noble gas scintillators, provided
the user supplies the required data inputs.

Two more refinements in the generation of optical photons are that the 
scintillation process takes into account a finite light emission rise-time, 
and that the Cerenkov photon origin density along the track segment is no longer 
constant, but assumes a linear decrease in the mean number of photons generated 
over the step as the radiating particle slows down.

For the propagation of optical photons, the reflectivity from a metal surface
may now be calculated by using a complex index of refraction \cite{embib:Iowa}.
%\footnote{Thanks to Sehwook Lee and John Hauptman (Iowa State 
%University)} 
Mie scattering was added following the \mbox{Henyey-Greenstein} approximation, 
with the forward and backward angles treated separately \cite{embib:Mie}.
% \footnote{Thanks to Xin Qian (Caltech) based on work from Vlasios 
% Vasileiou (University of Maryland)} 
Surface reflections may be simulated using look-up tables containing measured
optical reflectance for a variety of surface treatments \cite{em:opt2}. It is 
possible to define anti-reflective coatings, and transmission of a dichroic 
filter where the transmission, or conversely the reflection, is dependent on 
wavelength and incident angle.  The optical boundary process also works in 
situations where the surface is between two volumes, each defined in a different
parallel world, thus allowing optical photon propagation for previously 
impossible geometry implementations.

%end
