%%%%%%%%%%%%%%%%%%%%%%%%%%%%%%%%%%%%%%%%%%%%%%%%%%%
% evolution.tex
%%%%%%%%%%%%%%%%%%%%%%%%%%%%%%%%%%%%%%%%%%%%%%%%%%%
\label{sec:evolution}

A major trend in modern experimental science is the increased reliance upon
simulation to design complex detectors and interpret the data they produce.
Indeed, simulation has become mission-critical in fields such as high energy
physics and space science.  Another trend is the rapid increase in computing
power due to faster processors, multi-core architectures and distributed 
computing.  At the same time, advances in memory technology have not kept
pace and the amount of memory available per CPU cycle has decreased.  These 
trends have led to a re-examination of some of the basic assumptions of 
simulation computing, and to the evolution of the \Gfour{} simulation toolkit.

The toolkit approach has enabled \Gfour{} to serve a wide variety of user 
communities and to change as users' needs change.  Now sixteen years since its
initial public release, \Gfour{} continues to be the simulation engine of choice 
for high energy physics experiments at the LHC.  ESA and NASA have used and 
continue to use \Gfour{} in the design of spacecraft and the estimation of 
radiation doses received by astronauts and electronic components.  It is also 
used extensively in medical physics applications such as particle beam therapy, 
microdosimetry and radioprotection.  The basic extensibility of the toolkit 
has facilitated its expansion into new user domains, such as biochemistry,
material science and non-destructive scanning.   

Common to nearly all these domains, but especially true for high energy physics,
is the demand for increasingly detailed geometries and more sophisticated 
physical models.  This in turn drives the need for more CPU cycles, and the 
relative decrease of memory drives the need for more efficient memory 
management.  

It became clear that \Gfour{} could meet these challenges by adopting the 
multithreading approach and exploiting the multi-core architectures that have 
now become commonplace.  While significant effort went into the implementation 
of multithreading, the object-oriented design of the toolkit made the changes 
much less intrusive than might have been expected.  Section 2 of this document
will discuss the details and performance of this implementation.

The remainder of the document will deal with other improvements in the toolkit
since the last \Gfour{} general paper \cite{bib:generalpaper2}.
Section 3 covers improvements in kernel functionalities.  Among these are
new tracking and scoring capabilities, improved geometry models which have 
resulted in faster, more versatile experimental representations, and improved 
visualization techniques which provide users with more powerful ways to view 
them.  Section 4 provides updates in electromagnetic and hadronic physics 
modeling with discussions of recently introduced models, improvements in 
existing models and physics lists, and results from comparisons to data.
Extensions of the toolkit, including a new generic biasing framework, reverse
Monte Carlo, native analysis capability, and improved examples, are covered 
in Section 5.  Section 6 describes the extensive improvements in testing and
validation.  Modern web interfaces and standard testing tools have made it 
easier for users and developers alike to evaluate the performance of \Gfour{}.
The adoption of modern build tools addressed the need for flexible 
configuration and support on various computing platforms, as well as the 
ever-increasing number of data files needed by physics models.
This document concludes in Section 7 with a brief summary of \Gfour{} progress
and a discussion of prospects for the next decade.

\textit{A primer of terms.}
A number of terms within \Gfour{} have meanings which differ somewhat from 
general usage.  Although defined elsewhere \cite{bib:generalpaper2}, they are
reviewed here for convenience.

A \textit{track} is a snapshot of a particle at a particular point along its
path.  Instances of the class \gclass{G4Track} contain the particle's current
energy, momentum, position, time and so on, as well as its mass, charge,
lifetime and other quantities.

A \textit{trajectory} is a collection of track snapshots along the particle
path.

A \textit{step} consists of the two endpoints which bound the fundamental
propagation unit in space or time.  The length between the two points is chosen
by a combination of transportation and physics processes, and may be limited
to a fixed size by the user in cases where small step lengths are desired.  An
instance of \gclass{G4Step} stores the change in track properties between the 
two endpoints.

\textit{Process} has two meanings in \Gfour{}.  In the usual computer science 
sense, it refers to an instance of an application which is being executed.  This
is the meaning assumed in the discussion of multithreading.  In all 
other discussions the narrow \Gfour{} meaning is assumed: a class implementing a
physical or navigational interaction.  A \Gfour{} process is categorized by when 
the interaction occurs, either at the end of the step (\gclass{PostStep}) or 
during the step (\gclass{AlongStep}).

An \textit{event} consists of the decay or interaction of a primary particle and
a target, and all subsequent interactions, produced particles and four-vectors.
\gclass{G4Event} objects contain primary vertices and particles, and may contain
hits, digitizations and trajectories.

A \textit{run} consists of a series of events.

