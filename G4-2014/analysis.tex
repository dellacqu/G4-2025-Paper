%%%%%%%%%%%%%%%%%%%%%%%%%%%%%%%%%%%%%%%%%%%%%%%%%%%
% analysis.tex
%%%%%%%%%%%%%%%%%%%%%%%%%%%%%%%%%%%%%%%%%%%%%%%%%%%
\label{sec:analysis}

\subsection{\textbf{Analysis}}

\subsubsection{Introduction}
% History (AIDA based analysis)
% g4tools and new Geant4 analysis category
Analysis tools based on AIDA (Abstract Interfaces for Data Analysis) 
\cite{analysis:AIDA} have been used in \Gfour{} examples since release 3.0 in
December 2000, but until 2010 no analysis code was provided in the \Gfour{} 
source.  Several AIDA-compliant tools are available, including JAS, iAIDA, 
Open Scientist Lab and rAIDA \cite{bib:AppDevGuide} .
%  \cite{analysis:ApplicationDeveloperGuide}.
However some of them have not been maintained, some do not implement the AIDA
interfaces completely and some are not always easy to install and use.

A new analysis package based on g4tools~\cite{analysis:tools} was added in
\Gfour{} release 9.5 with the aim of providing users with a ``light-weight'' 
analysis tool available as part of the \Gfour{} installation without the need to 
link to an external analysis package.  It consists of the analysis manager 
classes and also includes the g4tools package.

g4tools provides code to write histograms and ntuples in several formats: 
ROOT~\cite{analysis:Root}, XML 2040 AIDA~\cite{analysis:AIDA} and CSV 
(comma-separated values) for ntuples and HBOOK~\cite{analysis:HBOOK}. It
is a part of the highly portable inlib and exlib~\cite{analysis:tools} 
libraries, which also include other facilities like fitting and plotting.  These
libraries are used to build the ioda application, available on all interactive
platforms so far, including iOS and Android, and able to read the file formats 
written with g4tools.   

The analysis classes provide a uniform, user-friendly interface to g4tools and 
hide from the user the differences between various output technologies.  They 
take care of higher level management of the g4tools objects (files, histograms
and ntuples), handle allocation and removal of the objects in memory and provide
the methods to access them via indexes.  For simplicity of use, all user 
interface analysis functions are provided within a single class which is seen by
the user as \gclass{G4AnalysisManager}.  Internally, this type is defined using
a \verb"typedef" and it can point to one of four type-specific output manager 
classes: \gclass{G4}\emph{Type}\gclass{AnalysisManager} where \emph{Type} can be 
\emph{Csv}, \emph{Root}, \emph{Xml} or \emph{Hbook}.

\subsubsection{Histogramming}
At present, one-dimensional and two-dimensional histograms are supported.  The 
properties of histograms already created can be changed with the use of
dedicated \texttt{Set()} functions including a limited set of parameters for 
histogram plotting. 

\gclass{G4AnalysisManager} also provides extensions to g4tools suitable for 
\Gfour{} applications.  Users can choose units and functions which are then 
automatically applied to filled values, a binning scheme (linear or logarithmic)
and an ASCII option which activates the printing of histograms to an ASCII file.
The activation option allows the user to activate only selected histograms.  
When this option is selected, only the histograms marked as activated are 
returned, filled or saved in a file. 

Histograms may also be created and modified interactively or in a macro 
using the rich set of commands defined in the \gclass{G4AnalysisMessenger} class.

\subsubsection{Ntuples}
Ntuples with \emph{int}, \emph{float} and \emph{double} column types are 
supported.  Depending on the selected output format, more files can be 
generated when more than one ntuple is defined in a user application.  This is
the case for XML and CSV, which do not allow writing more than one ntuple to a
file.  The ntuple file name is then generated automatically from the base file
name and the ntuple name.

\subsubsection{Multithreading}
Like all other \Gfour{} components, the analysis code was adapted for 
multithreading.  In multithreading mode, instances of the analysis manager 
are internally created on the master and worker threads and data accounting is 
processed in parallel on worker threads.  The migration to multithreading 
requires no changes in the user's client analysis code.  HBOOK output is not 
supported in multithreading mode.

Histograms produced on worker threads are automatically merged on the call to 
\verb"Write()" and the result is written to a master file.  Ntuples produced on
worker threads are written to separate files, the names of which are generated 
automatically from a base file name, a thread identifier and eventually also an
ntuple name.  No merging of ntuples is performed in order to avoid an associated 
time penalty. 

