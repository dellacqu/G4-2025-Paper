%%%%%%%%%%%%%%%%%%%%%%%%%%%%%%%%%%%%%%%%%%%%%%%%%%%
% barashenkov.tex
% Author: Vladimir Grichine
%%%%%%%%%%%%%%%%%%%%%%%%%%%%%%%%%%%%%%%%%%%%%%%%%%%
% \noindent {\emph{Barashenkov cross sections}}
\paragraph{Barashenkov cross sections}
The Barashenkov data set describes proton, neutron and charged pion cross 
sections (total and inelastic) on nuclei~\cite{hadbib:bar90,hadbib:bar89}. 
The Barashenkov interpolation for the total and inelastic cross sections is
essentially based on a quasi-optical model for high energies (T$\,\,>\,\,$2 GeV)
and on phenomenology, with correction terms of the form $\pi r_o A^{2/3}$, 
with $r_o\sim\,\,$1 fm.  The total, inelastic (and elastic) cross sections were
modeled with:
\[
\sigma(T,A)=\pi \left[r_{o}A^{1/3}+\lambda(T,A)\right]^{2}f(T)\phi(A)^{\alpha(T)},
\]
where $\lambda$ is the de Broglie length of the projectile in the center of 
mass system, $T$ is the kinetic energy of the projectile in the lab, $A$ is the 
atomic weight and $r_{o}\sim 1\,\,$fm. 
The functions $f(T)$, $\phi(A)$ and $\alpha(T)$ are series of the form:
\[
\sum_i a_i T^{b_i}\quad\textrm{or}\quad \sum_i a_i A^{b_i}.
\]
The general behavior of the optical models is to predict constant cross sections
for very high energies.  However, experimental data show a moderate relativistic 
rise of hadron-nucleus cross sections.  For this reason the Glauber model was
used to describe hadron-nucleus cross sections in the high energy region (above
90 GeV).
