%%%%%%%%%%%%%%%%%%%%%%%%%%%%%%%%%%%%%%%%%%%%%%%%%%%%%%%%%%%%%%%%%%%
% deexcitation.tex
% Authors: Vladimir Ivantchenko, Jose Manuel Quesada, Gunter Folger
%%%%%%%%%%%%%%%%%%%%%%%%%%%%%%%%%%%%%%%%%%%%%%%%%%%%%%%%%%%%%%%%%%%
\paragraph{Nuclear de-excitation models}
The final de-excitation of a nucleus to a thermalized state is simulated by
several semi-classical models which are responsible for sampling the 
multiplicity of neutrons, protons, light ions, isotopes, and photons.  They
are:
\begin{itemize}
\item Fermi break-up (FBU) \cite{hadbib:mfm-bondorf-1995},
\item statistical multifragmentation \cite{hadbib:mfm-bondorf-1995},
\item fission, based on the Bohr-Wheeler semi-classical model 
      \cite{hadbib:jinr-fis-1977, hadbib:inr-fis-1993},
\item evaporation of nucleons and light fragments, which is handled by models 
      based on either
\begin{itemize}
  \item the Weisskopf-Ewing model \cite{hadbib:weisskopf-1940} for fragments up
        to and including $\alpha$ particles, or
  \item the generalized evaporation model (GEM) \cite{hadbib:gem-2001} for the 
        emission of fragments with masses up to $^{28}$Mg, and 
\end{itemize}
\item \gclass{G4PhotonEvaporation}, which simulates the emission of  
\begin{itemize}
\item discrete gammas according to the E1, M1 and E2 transition probabilities
      taken from the PhotonEvaporation database, which in turn is based on 
      the Evaluated Nuclear Structure Data File (ENSDF) \cite{hadbib:ENSDF}, 
      and 
\item continuous gammas according to the E1 giant dipole resonance strength 
      distribution.
\end{itemize}
\end{itemize}

These models are managed by the \gclass{G4ExcitationHandler} class, in which
they may be invoked in complement or sometimes concurrently with each other.
Some of them have been thoroughly validated and have undergone continuous 
improvement in recent years \cite{hadbib:calor-2008, hadbib:pmb-fluka-g4-2011}.

In order to properly describe the double differential cross sections and isotope
production data of the IAEA spallation benchmark 
\cite{hadbib-iaea-spa-benchmark,hadbib:iaea-spa-2009}, the standard and GEM 
models were combined to form a hybrid model, and the fission model was improved
\cite{hadbib:pnst-preco-2011,hadbib:ijrb-space-2012,hadbib:iaea-spa-2009}.

For radiobiological applications it is essential that the FBU model be used by 
default for the de-excitation of light fragments (Z $<$ 9, A $<$ 17), taking 
into account Pauli blocking and all possible decay channels into stable and 
long-lived fragments.  Validations
\cite{hadbib:nima-pshenich-2010,hadbib:pmb-fluka-g4-2011}
triggered many of the refinements to this model.

For proton and ion beam therapy applications, the photon evaporation model, 
which is critical for the tracking of the Bragg peak from emitted prompt gammas, 
was improved \cite{hadbib:prompt-gamma-2014}.  The statistical 
multifragmentation model, responsible for the explosive break-up of heavier hot
nuclei  (Z $>$ 8, A $>$ 16, and excitation energy $>$ 3 MeV/u), is relevant 
in simulations of shielding from cosmic radiation and has also been validated
\cite{hadbib:ijrb-space-2012,hadbib:nima-pshenich-2010}. 

