%%%%%%%%%%%%%%%%%%%%%%%%%%%%%%%%%%%%%%%%%%%%%%%%%%%
% precompound.tex
% Author: Jose-Manuel Quesada
%%%%%%%%%%%%%%%%%%%%%%%%%%%%%%%%%%%%%%%%%%%%%%%%%%%
\paragraph{The precompound model}
The native \Gfour{} pre-equilibrium  model is based on a version of the 
semi-classical exciton model \cite{hadbib:gudima83} and is used as the back-end
stage of several cascade and quark-gluon string generators.  It handles the 
de-excitation of the remnant nucleus from its formation immediately following a
cascade or high energy collision, until it reaches equilibrium.  During this 
time, internal transitions of the pre-compound nuclear system compete with 
nucleon and light cluster emissions.  The passage to the state of statistical 
equilibrium, which happens when the transition probabilities for increasing and 
decreasing the exciton number become approximately equal (equilibrium condition),
is roughly characterized by an equilibrium number of excitons $n_{eq}$.  In the
simulation $n_{eq}$ is a calculated number based on the assumption that the 
equilibrium condition is met. 

Some refinements were introduced recently 
\cite{hadbib:calor-2008, hadbib:ijrb-space-2012, hadbib:iaea-spa-2009}, 
namely more realistic inverse cross section parameterizations and combinatorial
factors for particle emission, a phenomenological parameterization of the 
transition matrix elements, and a more physically consistent condition for the 
transition to equilibrium, since in certain circumstances this condition is 
reached well before the previously used rough estimate of $n_{eq}$.

At the end of the pre-equilibrium stage, the residual nucleus should be left in
an equilibrium state, in which the excitation energy is shared by the entire 
nuclear system.  Such an equilibrated compound nucleus is characterized by its 
mass, charge and excitation energy with no further memory of the steps which led
to its formation.

