%%%%%%%%%%%%%%%%%%%%%%%%%%%%%%%%%%%%%%%%%%%%%%%%%%%
% neutrons.tex
% Authors: Tatsumi Koi, Vladimir Ivantchenko
%%%%%%%%%%%%%%%%%%%%%%%%%%%%%%%%%%%%%%%%%%%%%%%%%%%
\paragraph{Low energy neutron models}
As of \Gfour{} 10.0, there were three models treating low energy neutrons with
high precision: NeutronHP, LEND and NeutronXS.  The NeutronHP models, for
elastic, inelastic, capture and fission reactions, have been part of \Gfour{}
for many years \cite{bib:G4}.  They depend on the \Gfour{} Neutron Data Library 
(G4NDL) for cross sections, branching ratios, final state multiplicities and
final state energy and angular distribution parameters.  In its original 
formulation, G4NDL data were drawn from nine different databases,
ENDF/B-VI \cite{hadbib:ENDF}, Brond-2.1 \cite{hadbib:BROND}, CENDL2.2
\cite{hadbib:CENDL}, EFF-3 \cite{hadbib:EFF}, FENDL/E2.0 \cite{hadbib:FENDL},
JEF2.2 \cite{hadbib:JEF}, JENDL-FF \cite{hadbib:JENDLFF}, JENDL-3 
\cite{hadbib:JENDL3} and MENDL-2 \cite{hadbib:MENDL}, 
with the majority coming from the Fusion Evaluated Nuclear Data Library (FENDL).  
This changed in \Gfour{} version 9.5 when G4NDL became solely dependent on US 
ENDF/B-VI and VII (Evalulated Nuclear Data Files) \cite{hadbib:ENDF}. Although 
the other databases are no longer used by G4NDL, they are still available for 
use with \Gfour{}.
  
Many evaluated data libraries, such as ENDF/B-VII.0, JEFF-3.1 and JENDL-4.0, 
have been converted to the \Gfour{} format \cite{hadbib:mendoza} and can be 
obtained at the IAEA Nuclear Data Services website \cite{hadbib:IAEA} .  
% https://www-nds.iaea.org/geant4/.  

NeutronHP was recently extended to include a new, detailed fission fragment 
generator (FFG) which was designed to model the complete detectable portion of
a fission event.  The event is modeled by taking into account mass and momentum 
conservation.  Fission products, from gammas to nuclear fragments are produced 
with realistic energies.  Ternary fission is
supported, even though its probability is low, but is currently limited to 
alpha particles.  The FFG is data-based, but designed to accommodate direct
physics models.  An example of this is symmetric fission and its angular 
dependencies.  This also allows the FFG to fission any isotope, provided that
the daughter product yield data are available. 

Because NeutronHP provides detailed cross sections in the resonance region and 
on-the-fly Doppler broadening, the code can be quite slow and for this reason 
is often not used in physics lists.  In order to provide improved CPU 
performance while retaining part of the NeutronHP precision, the NeutronXS 
elastic and capture cross sections were developed.  These cover neutron energies
from sub-eV up to the TeV range.  Below 20 MeV the detailed cross section data 
of the NeutronHP models are binned logarithmically and tabulated.  This preserves
most of the resonance structure.  At all energies the final state for elastic 
scattering and capture is generated using algorithms from the 
\gclass{G4ChipsElasticModel} and \gclass{G4NeutronRadCapture} models.  The 
NeutronXS cross sections yield a roughly four-fold decrease in CPU time for 
neutron propagation compared to the NeutronHP models and, as of release 10.0,
are now standard in most physics lists.

Another alternative to NeutronHP is the set of LEND (Livermore) neutron models.
These were designed to be faster than NeutronHP, with more streamlined code, but
to provide the same physics.  In the LEND models the cross sections are evaluated 
at a fixed temperature, rather than the on-demand temperature calculations of 
NeutronHP.  This results in a factor five increase in speed.  These models use
the Livermore online database for their reaction data.  It is based in part on 
ENDF/B-VII and can be obtained from the Lawrence Livermore National Laboratory
ftp site \cite{hadbib:llnlftp} .
%  ftp://gdo-nuclear.ucllnl.org/pub .\\

