%%%%%%%%%%%%%%%%%%%%%%%%%%%%%%%%%%%%%%%%%%%%%%%%%%%
% nucleusnucleusXS.tex
% Author: Vladimir Uzhinsky
%%%%%%%%%%%%%%%%%%%%%%%%%%%%%%%%%%%%%%%%%%%%%%%%%%%
\paragraph{Nucleus-nucleus cross sections}
The simulation of nucleus-nucleus interactions and the corresponding cross
sections is required by accelerator experiments, cosmic ray studies and 
medical applications, to name a few domains.

Because nuclei are charged, total and elastic cross sections are
infinite due to Coulomb interaction. In reality, they are restricted
by the screening of the atomic electrons. This interaction leads to a
small-angle scattering which can be ignored in a first approximation.
Thus, inelastic cross sections are the most important ones.
With increasing energy electromagnetic dissociation (EMD) becomes dominant,
especially for the collisions of heavy nuclei.  At low and intermediate energies
EMD does not play an essential role, while the nuclear break-up and 
multi-particle productions dominate.

%{\bf why this sentence is here if we don't use this\\
%There are good methods of EMD calculations
%\cite{hadbib:AAx1,hadbib:AAx2,hadbib:AAx3}, but we do not implement them
%because in low energy domain EMD results mainly in one or two neutron
%production.} 
The strong interaction cross sections can be calculated in the Glauber 
approximation \cite{hadbib:AntiA5,hadbib:AAx4} at high ($>$ 1 GeV) energies.
The description of the cross sections at low and intermediate energies is the
challenging component.

%{\bf All the next section is very confusing, it seems a historical excursus that does not clarify really what is used in G4,
%we should substantially revisit this\\
%Very simple expression was proposed in paper \cite{hadbib:AAx5} many years
%ago -- $\sigma_{AB}=\pi (R_A+R_B-c)^2$, where $R_A$ and $R_B$ are radii of
%$A$ and $B$ nuclei ($R_A=r_0\ A^{1/3}$). As it was found latter,
%$r_0\simeq$ 1.36 fm, and $c\sim$ 0 -- 1.5 fm which depend on a projectile
%energy. In papers \cite{hadbib:AAx6}, the following expression was proposed
%for $c$, $c=x(A^{-1/3}+B^{-1/3})$. It was additional improved in
%the paper \cite{hadbib:AAx7}.

%In order to extend the parameterization in the intermediate energy range,
%authors of the paper \cite{hadbib:AAx8} considered the expression:
%$\sigma_{AB}=\pi R_{int}^2\ (1-B/E_{CMS})$, where $R_{int}$ was subdivided
%into two parts, one was energy independent according to the author's analysis
%of experimental data, the second term had an energy dependence. $B$ is
%the Coulomb barrier of projectile-target system, and $E_{CMS}$ is CMS-energy
%of the system: $B=Z_A Z_B e^2/r_C(A^{1/3}+B^{1/3})$. At this, they obtained
%more reasonable value for $r_0$, $r_0=1.1$ fm. The parameterization was also
%applyed at low energies \cite{hadbib:AAx9} introdicing more parameters, and
%fitting experimental data. Recently, the last parameterization was validated for
%light nucleus projectiles \cite{hadbib:AAx10}. As it was shown there,
%the approach works rather well.\\
%Proposal:\\
%}
A first simple expression was proposed in \cite{hadbib:AAx5}:
$\sigma_{1,2}=\pi (R_1+R_2-c)^2$, where $R_1$ and $R_2$ are the radii of
the two interacting nuclei ($R=r_0\ A^{1/3}$), $r_0\simeq$ 1.36 fm, and 
$c\sim$ 0 -- 1.5 fm, depending on a projectile energy (following  
\cite{hadbib:AAx6} and the further refinements of \cite{hadbib:AAx7}
 $c \propto (A_1^{-1/3}+A_2^{-1/3})$).

In order to extend the parameterization to the intermediate energy range
\cite{hadbib:AAx8} $\sigma_{AB} = \pi R_{int}^2\ (1-B/E_{CMS})$ can be used,
where $R_{int}$ is composed of two terms, energy dependent and independent, 
$B = Z_A Z_B e^2/r_C(A^{1/3}+B^{1/3})$ is the Coulomb barrier of the 
projectile-target system, and $E_{CMS}$ is center-of-mass system energy.

In \Gfour{} the ``Sihver'', ``Kox'' and ``Shen'' parameterizations 
\cite{hadbib:AAx7,hadbib:AAx8,hadbib:AAx9} are used, with the Shen 
parameterization recommended for all physics lists.

