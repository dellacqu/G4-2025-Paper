%%%%%%%%%%%%%%%%%%%%%%%%%%%%%%%%%%%%%%%%%%%%%%%%%%%
% antibaryonXS.tex
% Authors: Aida Galoyan, Vladimir Uzhinsky
%%%%%%%%%%%%%%%%%%%%%%%%%%%%%%%%%%%%%%%%%%%%%%%%%%%
%{\bf Anti-baryon Cross Sections} \linebreak[4]
% \noindent {\emph{Antinucleus--nucleus cross sections}}
\paragraph{Antinucleus--nucleus cross sections}
Production of anti-nuclei, especially anti-$^4{\rm He}$, has been observed in
nucleus-nucleus and proton-proton collisions by the RHIC and LHC experiments. 
Contemporary and future experimental studies of anti-nucleus production require
a knowledge of anti-nucleus interaction cross sections with matter which are 
needed to estimate various experimental corrections, especially those due to 
particle losses which reduce the detected rate.  Because only a few measurements
of these cross sections exist, they were calculated using the Glauber approach 
\cite{hadbib:AntiA1,hadbib:AntiA2,hadbib:AntiA3} and the Monte Carlo averaging 
method proposed in \cite{hadbib:AntiA4,hadbib:AntiA5}.

Two main considerations are used in the calculations: a parameterization of the
amplitude of antinucleon-nucleon elastic scattering in the impact parameter
representation and a parameterization of one-particle nuclear densities for
various nuclei. The Gaussian form from \cite{hadbib:AntiA1,hadbib:AntiA3} was
used for the amplitude and for the nuclear density the Woods-Saxon distribution
for intermediate and heavy nuclei and the Gaussian form for light nuclei was 
used, with parameters from the paper \cite{hadbib:AntiA6}. Details of the 
calculations are presented in \cite{hadbib:AntiA7}.

Resulting calculations agree rather well with experimental data on anti-proton
interactions with light and heavy target nuclei ($\chi^2/NoF$ = 258/112) which 
corresponds to an accuracy of $\sim $8\% \cite{hadbib:AntiA7}.  Nearly all
available experimental data were analyzed to get this result.  The predicted 
antideuteron-nucleus cross sections are in agreement with the corresponding 
experimental data \cite{hadbib:AntiA8}.

Direct application of the Glauber approach in software packages like \Gfour{}
is ineffective due to the large number of numerical integrations required. To 
overcome this limitation, a parameterization of calculations 
\cite{hadbib:ggepjc,hadbib:ggnimb} was used, with expressions for the total 
and inelastic cross sections as proposed above in the discussion of the 
Glauber-Gribov extension. Fitting the calculated Glauber cross sections
yields the effective nuclear radii presented in the expressions for $\bar pA$,
$\bar dA$, $\bar tA$ and $\bar \alpha A$ interactions:
\\
$$
R^{eff}_A=a\ A^b\ + \ c/A^{1/3}.
$$
The quantities $a$, $b$ and $c$ are given in \cite{hadbib:AntiA7}.

As a result of these studies, the \Gfour{} toolkit can now simulate 
anti-nucleus interactions with matter for projectiles with momenta between 
100~MeV/c and 1~TeV/c per anti-nucleon.

