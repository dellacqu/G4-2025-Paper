%%%%%%%%%%%%%%%%%%%%%%%%%%%%%%%%%%%%%%%%%%%%%%%%%%%%%%%%%%%%%%%%%%%%
% cascade.tex
% Authors: Mike Kelsey, Davide Mancusi, Gunter Folger, Dennis Wright
%%%%%%%%%%%%%%%%%%%%%%%%%%%%%%%%%%%%%%%%%%%%%%%%%%%%%%%%%%%%%%%%%%%%
\paragraph{Intranuclear cascade models}
% \noindent {\emph{Intranuclear cascade models}}
Three intranuclear cascade models are now offered in \Gfour{}: Bertini, Binary
and INCL++.  The extended Bertini cascade \cite{hadbib:bert} is valid for
p, n, $\pi$, K, $\Lambda$, $\Sigma$, $\Xi$, $\Omega$ and $\gamma$ projectiles
with incident energies between 0 and 15 GeV.  It is also valid for captured 
$\mu^-$ , $K^-$ and $\Sigma^-$.  Recent extensions allow this model to be used
for cascades initiated by high energy muons and electrons.  Although this model
has its own precompound and deexcitation code, an option exists for using the 
native \Gfour{} precompound and deexcitation modules discussed in the following
section.
%  \cite{hadbib:pnst-preco-2011}.

The Binary cascade \cite{hadbib:binary} simulates p and n-induced cascades
below 10 GeV, and $\pi$-induced cascades below 1.3 GeV.  This is done by 
propagating hadrons in a smooth nuclear potential, and forming and decaying 
strong resonances to produce secondaries.  The model relies on the native 
\Gfour{} precompound and deexcitation code to handle the post-cascade steps. 

The Li\`ege Intranuclear Cascade model (\incl) \cite{hadbib:incl} has seen
extensive development since its introduction in \Gfour{}. The original Fortran 
model was completely redesigned and rewritten in C++ and is now known as 
\inclxx\ \cite{hadbib:inclxx}. It extends the applicability of the legacy 
version up to $\sim15$~GeV incident energy, while remaining physics-wise 
equivalent for nucleon- and pion-induced reactions below 1~GeV. In addition,
\inclxx\ has been extended to handle reactions induced by light ions up to
$A=18$.  By default, \inclxx\ uses the \Gfour{} native de-excitation immediately
after the cascade stage; it does not include an intermediate pre-equilibrium 
step.  Coupling to the \abla\ de-excitation model \cite{hadbib:ablav3} is also
possible.

% Quesada, J.M., Ivantchenko V., Ivantchenko A., Cortes M.A., Folger G.,
% Howard A., Wright D. Recent Developments in Pre-Equilibrium and 
% De-excitation Models in Geant4. Progress in Nuclear Science and Technology. 
% Proceedings of the Joint International Conference Supercomputing in Nuclear 
% Applications and Monte Carlo 2010. October 17-21. Tokyo, Japan.
