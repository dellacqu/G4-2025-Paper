%%%%%%%%%%%%%%%%%%%%%%%%%%%%%%%%%%%%%%%%%%%%%%%%%%%
% stopping.tex
% Authors: Julia Yarba, Mike Kelsey, Krzysztof Genser
%%%%%%%%%%%%%%%%%%%%%%%%%%%%%%%%%%%%%%%%%%%%%%%%%%%
\paragraph{Stopping models}
The \Gfour{} toolkit includes processes to simulate the stopping and capture of
hadrons and heavy leptons on nuclei.  Previous parameterized models for $\pi^-$
and $K^-$ capture were replaced by the \Gfour{} Bertini intranuclear cascade for 
negative mesons, baryons and muons, and with the FTF string model for antibaryon
capture and annihilation.  

The Bertini cascade model handles the capture process by selecting a random
location within the nucleus, weighted by the radial nucelon density, to initiate
the cascade process.  The subsequent cascade is propagated using the same code 
used for any other hadron-nucleus interaction.  In the FTF model, the antibaryon
annihilates with a nucleon near the outer ``surface'' of the nucleus, producing 
a small number of pions.  Those secondaries and the excited nucleus are passed 
either to the \Gfour{} de-excitation model (invoked by \gclass{G4PrecompoundModel}),
or to one of the cascade models (Bertini or Binary) for final disposition 
depending on the energy.

Capture of negative muons on nuclei is also handled using the Bertini model.
The muon capture process deals with the atomic capture, the subsequent 
electromagnetic cascade of the muon down to the lowest orbit, including
photon and Auger electron emissions, and the decision about whether the
bound muon should decay in that orbit or be captured into the nucleus.  If
the latter, the Bertini cascade selects a random location within the nucleus 
(weighted as above), where the $\mu^-$ interacts with a proton, producing a 
neutron, neutrino, and one or more radiative photons.  Note that in \Gfour{} 
release 10.0 onward, the radiative cross sections have been set to zero.  As 
above, the neutron is treated as a cascade secondary and processed using the 
existing code of the Bertini model.

