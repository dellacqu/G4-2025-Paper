%%%%%%%%%%%%%%%%%%%%%%%%%%%%%%%%%%%%%%%%%%%%%%%%%%%%%
% nucleusnucleus.tex
% Authors: Davide Mancusi, Dennis Wright
%%%%%%%%%%%%%%%%%%%%%%%%%%%%%%%%%%%%%%%%%%%%%%%%%%%%%
\paragraph{Nucleus-nucleus models}
As of release 10.0 there were six \Gfour{} models capable of handling
nucleus-nucleus collisions: binary light ion, abrasion/ablation, 
electromagnetic dissociation, QMD, INCL++ and FTF models.

The Binary Light Ion model handles collisions in which either the projectile or
the target has mass $A < 13$.  Based on the \Gfour{} Binary Cascade model 
\cite{hadbib:binary}, it is valid above 80 MeV and below 10 GeV/nucleon. 

Operating over a similar energy range, but without limits on the projectile or
target masses, the \gclass{G4WilsonAbrasion} model, based on NUCFRG2 
\cite{hadbib:wilson} is faster, but less detailed, than the Binary Light Ion 
model.  It is a geometrical model in which a portion of the target nucleus along
the incident path of the projectile is gouged out, forming a forward-going 
compound nucleus and a residual target.  The associated Wilson ablation model is
used to de-excite the products of the initial collision.

Also based on NUCFRG2, \gclass{G4EMDissociation} is an electromagnetic 
dissociation model provided to handle the production of nuclear fragments 
resulting from the exchange of virtual photons between projectile and target 
nuclei.  This model is valid for nuclei of all masses and all energies.
  
QMD (Quantum Molecular Dynamics) is a native \Gfour{} model based on an extension
of the classical molecular dynamics model introduced in release 9.1.  Each 
nucleon in the target and projectile nuclei is treated as a gaussian wave packet
which propagates with scattering through the nuclear medium, taking Pauli 
exclusion into account.  The nuclear potential is that of two merging nuclei and
its shape is re-calculated at each time step of the collision.  
Participant-participant scattering is also taken into account.  
These last two facts combine to make the model rather slow for collisions of
heavy nuclei, but the production of nuclear fragments versus energy is well 
reproduced.  The model is valid for all projectile-target combinations and for
projectile energies between 100 MeV/nucleon and 10 GeV/nucleon.  Since its
introduction, the model was made Lorentz covariant and improvements were made in
fragment prodcution at relativistic energies.
 
The \inclxx\ model, covered above, can also accommodate nucleus-nucleus 
reactions provided the projectile has a mass below $A = 19$ and an energy 
between 1 MeV/nucleon and 3 GeV/nucleon.  A broad validation campaign on 
heterogeneous observables has shown that, in spite of the conceptual 
difficulties, the extended \inclxx\ model yields predictions in fair agreement 
with experimental data; it is however crucial to make a suitable choice for the 
coupling with the statistical de-excitation model.

The FTF model, covered above, is capable of modeling 
reactions with all combinations of projectile and target mass, with projectile 
energies in the range 2 GeV/nucleon to about 1 TeV/nucleon.  However, validation
of this application is still in progress, and collisions of two heavy nuclei are 
expected to be computationally expensive.

