%%%%%%%%%%%%%%%%%%%%%%%%%%%%%%%%%%%%%%%%%%%%%%%%%%%
% gamleptnuclear.tex
% Authors: Dennis Wright, Mike Kelsey
%%%%%%%%%%%%%%%%%%%%%%%%%%%%%%%%%%%%%%%%%%%%%%%%%%%
\paragraph{Gamma- and lepto-nuclear models}
Due to the relatively small electromagnetic coupling, gamma- and lepto-nuclear
reactions play a small role in high energy physics calorimetry. They are 
important, though, for nuclear, medium energy and cosmic ray physics.  For this 
reason \Gfour{} models for these reactions were extended and improved.  

The \gclass{G4PhotoNuclearProcess} is implemented by two models, the Bertini 
cascade below 3.5 GeV and the Quark-Gluon-String (QGS) model above 3 GeV.  Both
models treat the incident gamma as if it were a hadron interacting with a 
nucleon within the nuclear medium.  Nevertheless, below 30 MeV the Bertini model
does capture some collective nuclear effects such as the giant dipole resonance.

Both the electro-nuclear and muon-nuclear models
(\gclass{G4ElectroVD\allowbreak{}Nuclear\allowbreak{}Model}
and \gclass{G4MuonVD\allowbreak{}Nuclear\allowbreak{}Model}) exploit two
features of the hybrid electromagnetic hadronic interaction: the factorization
of the interaction into separate hadronic and electromagnetic parts and the 
treatment of the exchanged 
photon as if it were a hadron.  The electromagnetic vertex produces a virtual 
photon from a two-dimensional cross section table and uses the method of 
equivalent photons to make the photon real.  As in the photo-nuclear case 
mentioned above, the photon is then treated as a hadron for the remainder of the
interaction.  For real photons below 10 GeV the Bertini cascade handles the 
interaction;  above 10 GeV the photon is converted to a neutral pion and the 
interaction proceeeds using the FTF string model.

