%%%%%%%%%%%%%%%%%%%%%%%%%%%%%%%%%%%%%%%%%%%%%%%%%%%
% radioactivedecay.tex
% Author: Dennis Wright
%%%%%%%%%%%%%%%%%%%%%%%%%%%%%%%%%%%%%%%%%%%%%%%%%%%
\paragraph{The radioactive decay process}
The \gclass{G4RadioactiveDecay} process and model handles $\alpha$, $\beta^-$,
$\beta^+$, isomeric transition (IT) and electron capture (EC) decays, and can be
applied to generic ions both in flight and at rest.   

Details for each decay or level transition, such as nuclear level energies, 
branching ratios and reaction Q values, come from the \Gfour{} RadioactiveDecay 
database, which currently contains entries for 2798 nuclides.  Details of 
specific gamma levels used for IT decays are taken from the \Gfour{} 
PhotonEvaporation database.  Both the PhotonEvaporation and RadioactiveDecay 
databases take their data from the Evaluated Nuclear Structure Data File (ENSDF)
\cite{hadbib:ENSDF} and have recently been rationalized so that their common 
nuclear levels have identical values.

Beginning with \Gfour{} release 9.6 and continuing through releases currently in
preparation, a number of improvments have been made to the radioactive decay 
package.  These include:

\begin{itemize}
\item a complete review of the PhotonEvaporation and RadioactiveDecay databases,
      and updating to the 2013 version of ENSDF,
\item the ability to model decays with lifetimes as short as 1 ps,
\item decays of observationally stable ground states, that is, those having
      very long predicted life times, but which have not yet been observed to 
      decay,  
\item the addition of unique first, second and third forbidden $\beta^-$ and
      $\beta^+$ decays,
\item the default invocation of the atomic relaxation model after IT and EC 
      decays, and 
\item improved energy conservation for all decay modes.
\end{itemize}

