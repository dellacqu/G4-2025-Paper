%%%%%%%%%%%%%%%%%%%%%%%%%%%%%%%%%%%%%%%%%%%%%%%%%%%
% testingtools.tex
%%%%%%%%%%%%%%%%%%%%%%%%%%%%%%%%%%%%%%%%%%%%%%%%%%%

\subsubsection{Testing tools}
Within the release process, integration of new or modified code into reference
tags is performed.  This requires regular nightly testing of new or modified 
code against the code base on a number of different platforms.  A platform is a 
specific combination of hardware and operating system version, a specific 
version of compiler, and a set of \Gfour{}-specific build options.  This regular
testing was migrated from a custom setup to using the CMake tool suite 
\cite{release:cdash}.  

There is a continuous testing cycle and nightly test runs.  Testing on a
specific platform is performed using the CTest tool of the CMake tool suite. 
For each platform, the source code corresponding to the set of tags to be tested
is checked out using SVN, CMake is run to configure and compile the code, and 
CTest runs the tests.  More than 80 unit test codes are applied, each testing a 
specific part of \Gfour{}.  In addition most examples and several benchmarks are
exercised during testing, bringing the total number of tests run to more than 
200. The exact number varies with platform, as optional software required by 
some tests may not be available on every platform.  CMake takes this into 
account by checking the availability of optional software during configuration, 
and skipping tests where software is missing.  Finally, a summary of test 
results is uploaded to the CDash web site.
  
In addition to nightly testing, continuous testing is performed on a small 
number of platforms using a restricted set of tests.  This regularly checks for 
changes in the status of tags, including also tags proposed for testing, and 
starts a test run when needed.  This restricted testing allows for an early,
but limited check of new software giving fast feedback to developers.
