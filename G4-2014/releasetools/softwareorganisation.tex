%%%%%%%%%%%%%%%%%%%%%%%%%%%%%%%%%%%%%%%%%%%%%%%%%%%
% softwareorganisation.tex
%%%%%%%%%%%%%%%%%%%%%%%%%%%%%%%%%%%%%%%%%%%%%%%%%%%

\subsubsection{Software organization}
\Gfour{} software is structured in a directory tree with each category 
corresponding to a high-level directory.  Categories such as geometry, which
is comprised of geometry shapes and navigation, and processes, which includes 
physics interaction models, are recursively subdivided into more and more 
specialized topics.  In general the C++ classes for each topic, such as a model
describing a specific type of interaction, are maintained or developed by one 
or a few developers. 

  Collaborative software revision management is handled by the SVN tool.  During
the development process, developers are encouraged to commit frequently changes 
to the trunk of the SVN repository.  When a change within a topic or directory 
is complete, the developer creates a tag of this code; features of the tag are 
documented changes in a History file, present in each sub(sub...) directory.  
A tag may be made of low-, intermediate- or high-level directories.   
Tags are recorded in the database using SVN commit hooks on the SVN server.  The
developer can propose such tags for inclusion into future releases using a web 
interface to the tags database.  Proposed tags will then enter the testing cycle. 
