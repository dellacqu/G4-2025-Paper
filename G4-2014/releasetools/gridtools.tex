%%%%%%%%%%%%%%%%%%%%%%%%%%%%%%%%%%%%%%%%%%%%%%%%%%%
% gridtools.tex
%%%%%%%%%%%%%%%%%%%%%%%%%%%%%%%%%%%%%%%%%%%%%%%%%%%

\subsubsection{Grid tools}
The simplified calorimeter is a \Gfour{} application which uses simplified 
versions of LHC calorimeters and produces simulation results for beam energies
between 1 and 500 GeV.  These results are then compared to normalized real data
so that the effects of changes between different \Gfour{} versions or physics
models on physics performance may be investigated.

Acquiring accurate simulation results requires running the simplified 
calorimeter for thousands of events.  This, in conjunction with the high 
energies, makes running the application on developer PCs or even local computing
clusters impossible due to time restrictions.  This need for computing power was
met by the use of resources provided by the Worldwide LHC Computing Grid 
\cite{GT:LHCG}.  
Simplified calorimeter jobs are split into smaller fragments and sent for 
execution in various grid sites around the world.  Results are then collected 
and pushed to the validation database.  The end user, typically a developer, is 
able to run on-demand analysis tasks through an interactive web application and 
produce plots.  The current grid tools implementation is based on the DIRAC 
workload management system which is used to distribute the jobs.  A system was 
developed to support different \Gfour{} tests, examples and applications that 
are or will be executed on the grid.  \Gfour{} libraries and executables are 
made available to the sites through the \Gfour{} CernVM-FS \cite{GT:CVMFS} 
repository.  Finally ROOT is used for the results format, merging of split jobs
and on-demand analysis.

