%%%%%%%%%%%%%%%%%%%%%%%%%%%%%%%%%%%%%%%%%%%%%%%%%%%
% qualityassurance.tex
%%%%%%%%%%%%%%%%%%%%%%%%%%%%%%%%%%%%%%%%%%%%%%%%%%%

\subsubsection{Quality assurance}
Each software release of \Gfour{} is validated against memory management issues,
in particular run-time memory errors (overwrites, memory corruption, etc.) and 
memory leaks, through use of the Valgrind~\cite{QA:valgrind} tool.  Verification
for run-time memory errors is completely automated as part of the system 
testing nightly builds, so that results for each test included in the testing 
suite can be retrieved directly from the testing dashboard and monitored along 
with the normal development process.  Specific sessions of Valgrind runs are 
executed periodically on a selected set of tests covering a rather wide range of
physics configurations to verify the absence of memory leaks during the 
execution of a simulation run for multiple simulated events; a summary of the 
results is distributed to developers by the release coordinator, together with 
hints on where problems may originate.  Memory leak checks are usually performed
on candidate releases as part of the software release process.

The Valgrind tool is also used for detecting errors in multithreaded simulation
applications, by means of its DRD~\cite{QA:DRD} module.  DRD allows the
identification of potential data-races happening during run-time among threads,
when one or more threads try to access the same memory location without proper
locking or protection.  It also allows easy identification and debugging of
lock contention cases, when a test program may get stuck in execution.

Static code analysis is regularly applied at every development release, by 
instrumenting the entire source code with the Coverity~\cite{QA:coverity}
tool.  The list of defects identified by the tool are first analysed and then
assigned to the responsible developers for further inspection and correction.
Since 2010, when this tool was first adopted for quality assurance, almost 3000
defects (most of them not critical) have been triaged and fixed, with priority
given to cases of potentially higher impact.

