%%%%%%%%%%%%%%%%%%%%%%%%%%%%%%%%%%%%%%%%%%%%%%%%%%%
% examples.tex
%%%%%%%%%%%%%%%%%%%%%%%%%%%%%%%%%%%%%%%%%%%%%%%%%%%
\label{sec:basicexamples}
\subsection{\textbf{Basic Examples}}
The \Gfour{} toolkit includes several fully coded examples which demonstrate the 
implementation of the user classes required to build a customized simulation.
The previous ``novice'' set of examples, oriented to beginning users, was 
refactored into ``basic'' and ``extended'' example sets in \Gfour{} 10.0.

The new ``basic'' examples cover the most typical use-cases of a \Gfour{} 
application while maintaining simplicity and ease of use.  They are provided as 
starting points for new application developers.  There are currently five such
examples, some of which include several options or sub-examples.  The features
demonstrated in each example will be presented in the following subsections.

All basic examples have been migrated to multithreading (MT) and no special 
steps are required to build them in this mode.  They will automatically run as
MT when they are built on the \Gfour{} libraries built with MT mode activated; 
otherwise they will run in sequential mode.  MT mode may be chosen by creating 
\gclass{G4MTRunManager} instead of \gclass{G4RunManager} in the \gclass{main()}
of the example.

% If the example is invoked in interactive mode, that is, without specifying a
% macro file on the command line, an interactive session will be established
% depending on the build options and \verb"~/.g4session".  For more details on
% this see the Application Developers' Guide \cite{bib:AppDevGuide}.  The most 
% advanced session possible will be instantiated.

% All basic examples include a macro \verb"vis.mac" that is invoked in interactive
% mode if the user builds with visualization.  It establishes a default viewer, 
% the most advanced available with the chosen session, a viewing angle, etc., 
% suitable for each example. In particular \verb"B1/vis.mac" demonstrates several
% features of the visualization system, namely the viewing style, the type and 
% appearance of trajectories, the ability to add axes, text, date and time, event
% number, the \Gfour{} logo and the ability to set the visibility of individual 
% volumes.  These are just a few of the features available with the \Gfour{} 
% visualization system \cite{bib:visprogress}.  A comprehensive list, with 
% guidance, can be found in Chapter 7 of the Application Developers Guide 
% \cite{bib:AppDevGuide}.

Basic examples can be run in interactive mode with visualization, or in 
batch mode.  The most suitable visualization parameters, such as a viewing 
angle, are selected for each example in order to promote a better understanding
of the example scenario and to demonstrate the most useful features of the 
visualization system.  A comprehensive list of visualization features can be 
found in Chapter 7 of the Application Developer Guide \cite{bib:AppDevGuide}.

\subsubsection{Example B1}
\verb"examples/basic/B1" demonstrates a simple application in which user actions
control the accounting of the energy deposit in volumes.  The dose in a selected 
volume is then calculated.  The geometry setup is defined with the use of 
simple placements (\gclass{G4PVPlacement}) and the physics is defined with the 
use of the QBBC pre-packaged physics list.  Scoring is implemented directly 
in the user action classes and an associated ``run'' object (\verb"B1Run")
is created.

This example also demostrates several features of the visualization system not
shown in other examples, namely the type and appearance of trajectories, the 
ability to add axes, text, date and time, event number, the \Gfour{} logo and
the ability to set the visibility of individual volumes.
  
\subsubsection{Example B2}
\verb"examples/basic/B2" simulates a simplified fixed target experiment. Two 
geometry setup options are provided: one using simple placements 
(\gclass{G4PVPlacement}) and one using parameterized volumes 
(\gclass{G4PVParameterisation}).  In addition a global, uniform, transverse 
magnetic field can be applied using 
\gclass{G4Global\allowbreak{}MagField\allowbreak{}Messenger}.  The physics setup
is defined using the FTFP\_BERT pre-packaged physics list with a step limiter.
Scoring is implemented with sensitive detectors and hits.

\subsubsection{Example B3} 
\verb"examples/basic/B3" simulates a simplified positron emission tomography
system.  The geometry is defined with simple placements and rotations.  A 
modular physics list is used.  Primary particles are $^{18}$F ions randomly 
distributed within a volume inside a simulated patient, with an associated 
radioactive decay process.  Scoring is implemented with \Gfour{} primitive 
scorers.

\subsubsection{Example B4}
\verb"examples/basic/B4" simulates a sampling calorimeter.  The geometry is 
defined using simple placements and replicas (\gclass{G4PVReplica}).  A global,
uniform, transverse magnetic field can be applied using 
\gclass{G4Global\allowbreak{}MagField\allowbreak{}Messenger}.
Physics is defined using the FTFP\_BERT reference physics list.

Energy deposits and track lengths of the charged particles are recorded event
by event in the absorber and gap layers of the calorimeter.  This example 
demonstrates four different scoring methods: user actions, user-developed 
objects, sensitive detectors and hits, and primitive scorers.

Also demonstrated is the use of \Gfour{} analysis tools for accumulating
statistics and computing the dispersion of the energy deposit and track lengths 
of the charged particles.  The resulting one-dimensional histograms and an 
ntuple are saved in an output file, which has ROOT format by default, but can
be changed at compilation time to other formats supported by g4tools (AIDA XML,
CSV for ntuples, or HBOOK). 

\subsubsection{Example B5}
\verb"examples/basic/B5" simulates a double-arm spectrometer with wire chambers,
hodoscopes and calorimeters. The geometry setup, less trivial than in previous 
examples, is defined using placements, rotations, replicas and a parameterization.
In addition a global, uniform, transverse magnetic field can be applied using 
\gclass{G4GlobalMagFieldMessenger}.  The physics setup is defined with the 
FTFP\_BERT reference physics list with a step limiter.  Scoring within wire 
chambers, hodoscopes and calorimeters is implemented with sensitive detectors 
and hits. 

\gclass{G4GenericMessenger} is used to define user interface commands specific 
to the user application and \Gfour{} analysis tools are used to output scoring
quantities in one-dimensional and two-dimensional histograms and an ntuple. 
